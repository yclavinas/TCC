%TODO: add contribuitions and future works
\chapter{Conclusão}\label{chapter8}
Este projeto apresentou uma proposta de desenvolvimento de um modelo de previsão de probabilidades elaborado a partir de uma implementação simples de algoritmos genéticos. Foi possível perceber uma evolução do modelo em relação ao modelo completamente aleatório, caracterizado pela população inicial.\\

Foram muitos os desafios enfrentados ao utilizar os testes propostos pelo RELM como função de {\it fitness} pelo algoritmo genético. O primeiro está relacionada ao tempo de execução e o uso desses a cada geração como função de {\it fitness}. Por se tratar de testes com uma grande quantidade de cálculos em grande quantidades de observações, o tempo gasto para analisar as informações poderia ser demasiado grande para ser viável continuar a utilizar os testes junto ao algoritmo. O segundo problema estava vinculado ao comportamento desses testes em uma aplicação de algoritmos genéticos, pois não havia conhecimento anterior sobre a aplicabilidade desses com Computação Evolutiva e se aplicável, se o resultado seria promissor.\\

Pelos resultados finais do L-test, foi verificado que tanto o primeiro desafio quanto o segundo, não foram suficientes para impossibilitar o uso dos testes e que as gerações de populações resultaram em indivíduos mais aptos para a previsão de sismos. Pelos resultados mostrados o trabalho mostra que aplicar Computação Evolutiva para prever ocorrências de sismos é minimamente promissor, uma vez que há uma evolução do modelo em relação ao modelo pseudo-aleatório.\\

Foi possível, portanto, vincular os testes proposto pelo RELM com algoritmos genéticos. A partir disso, agora, deve-se explorar as características da aplicação, tornar os testes implementados mais completos e estruturados, definir os operadores genéticos visando um comportamento adequado em relação aos indivíduos como para as populações evoluídas.\\

Posteriormente, o primeiro desafio foi minimizado, pelo uso de uma tabela em memória do fatorial. Alguns estudos deverão ser realizados para compreendermos melhor o porquê da alteração dos valores após a introdução da tabela.\\

O uso de operadores de peso adaptativos mostrou-se interessante e é provável que, sua inclusão na GA estudada traga ainda mais benefícios assim que outros problemas forem resolvidos. Problemas, tais quais, a independência dos {\it bins}, que influenciaram negativamente a performance da aplicação e serão alvos de maiores esforços.\\



Algumas propostas de melhorias e trabalhos futuros podem ser listadas:\\

• Compará-lo com um outro modelo de previsão de terremotos, como por exemplo, o {\it Relative Intensity} (RI). Para melhores efeitos de comparação, é interessante acrescentar cálculos de confiabilidade estatísca, como {\it p-value}, e criar um mapa demonstrativo da previsão relativa de cada área representada pelos {\it bins};\\%acho que tem no artigo 

• Implementar o operador de mutação específico para números reais e que tenha um comportamento direcionado a aplicação, capaz de alterar seus valores no decorrer das gerações capaz de equilibrar {\it explotaition} e {\it exploration};\\

• Aplicar alguns testes sugeridos por Zechar\citep{zechartheme} afim de definir a qualidade do modelo desenvolvido;\\

• Fazer analises de complexidade do algoritmo executado;\\

• Criar múltiplas observações para cada indivíduo para que o cálculo de incertezas possa ser aplicado e, assim, aumentarmos a qualidade dos modelos;\\

• Realizar experimentos em outras áreas do Japão, além de Kanto;\\

• Especializar a abordagem de algoritmo genético ({\it crossover} mais apropriado, mutação específica, ...) ou  utilizar soluções híbridas, entre algoritmo genéticos e outras técnicas de aprendizado de máquina;\\

• Incrementar os cálculos estatísticos com as ferramentas adequadas, desvio padrão, variância, etc, que compõe os cálculos dos testes demonstrados;\\

• A fim de evitar o {\it over fitting}, podemos inserir dados sismológicos, como sobre de magnitude e hora da ocorrência, aumentando a área de busca da aplicação;\\

• Cada {\it bin} é trato e considerado individualmente. Portanto a ocorrência de sismos em um {\it bin} não influência ocorrências de sismos em seus vizinhos, o que sabemos não ser verdade;\\

• Contrariando expectativas, execuções com operadores reais obtiveram desempenho pior que com operadores tradicionais pode estar relacionado ao fato de não existir influência entre os {\it bins}, uma vez que pode haver áreas de altíssima probabilidade próximas a áreas de baixa probabilidade.\\
