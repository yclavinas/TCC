\chapter{Computação Evolutiva e Previsão de Sismos}~\label{chapter4}
%TODO: get this info from main.pdf
%TODO: update this information
%A relação Computação Evolutiva e Previsão de Sismos ainda é escassa, pouco explorada. Esse capítulo é dedicado a mostrar o que já foi explorado dessa relação pela comunidade acadêmica. \\

O presente capítulo é dedicado a mostrar o que já foi explorado da relação entre Previsão de Sismos e Computação Evolutiva (EC) pela comunidade acadêmica.% Apesar de ser uma relação pouco estudada, a seguir descreveremos trabalhos correlatos.  \\

Uma das abordagens utilizada é a hibridização entre as técnicas de Computação Evolutiva (EC). Zhang e Wang \cite{Zhang2012} utilizaram Algoritmo Genético (GA) para refinar uma Redes Neurais (ANN) e, a partir dessa aplicação, criar um modelo de previsão. Zhou and Zhu \cite{zhou2014earthquake}, para realizar uma previsão da magnitude de sismos, fizeram uma combinação entre ANN e EC.\\

Muitas das aplicações estimam características dos sismos ou de suas atividades, como por exemplo calcular o {\it Peak Ground Acceleration} (PGA), \cite{pga_Kerh, Kermani2009, Cabalar2009}. PGA é uma medida da aceleração do sismo no solo e pode ser utilizada, por exemplo, para projetar estruturas mais resistentes a abalos sísmicos, tendo importância elevada em áreas próximas ao cinturão sísmico \cite{Cabalar2009}.\\

GA já foi utilizado também para decidir a localização, baseado em atividades sísmicas, de estações de sensoriamento no México \cite{Ramos2011}. Ele foi utilizado como uma ferramenta de projeto para construir uma rede de estações em diferentes regiões do México, objetivando formar uma rede com estações em locais ótimos, afim de alertar a população o mais rápido possível para evitar maiores desastres.\\

Já Nicknam \cite{Nicknam2010} e Kennett e Sambrigde \cite{Kennett1992} utilizaram EC para determinar parâmetros para modelos de falhas (como epicentro, localização, profundidade, etc.) de um dado sismo.\\%detalhar mais coisas!!!!!

Huda e Santosa \cite{ijse5762} recentemente publicaram um artigo em que buscam determinar, com algoritmos genéticos, a velocidade das ondas P e S no manto e na crosta terrestres. Ondas P são indicadas como a primeira falha encontrada em dados sismológicos e ondas S são as mudanças causadas na fase das ondas P \cite{ijse5762}. Essa pesquisa busca obter a estrutura do subsolo japonês e, geograficamente, possui o mesmo foco que a presente pesquisa.


In this section we will briefly discuss some reports of the
application of Evolutionary Computation and related method for
Earthquake Risk Analysis.

The usage of Evolutionary Computation in the field of earthquake risk
models is somewhat sporadic. Zhang and Wang~\cite{Zhang2012} used
Genetic Algorithms to fine tune an Artificial Neural Network (ANN) and
use this system to produce a forecast model. Zhou and
Zu~\cite{Feiyan2014} also proposed a combination of ANN and EC, but
their system only forecasts the magnitude parameter of
earthquakes. Sadat, in the paper~\cite{sadat2015application}, follows
the idea oF Zhou and Zu, aiming to predict the magnitude of the
earthquakes in North Iran, but in this case, he used ANN and GA.

Some sismological models were developed aiming to estimate parameter
values by using Evolutionarry Computation. For example, Evolutionary
Computation was used to estimate the peak ground acceleration of
seismically active
areas~\cite{Kermani2009,Cabalar2009,Kerh2010,Kerh2015}. Ramos~\cite{Ramos2011}
used Genetic Algorithms to decide the location of sensing stations and
Saeidian~\cite{saeidian2016evaluation} made a comparation in
performance between the GA and Bees Algorithm to decide which of those
techniques would performe better when chosing the location of sensing
stations. Nicknam et al.~\cite{Nicknam2010} and Kennett and
Sambridge~\cite{Kennett1992} used evolutionary computation to
determine the Fault Model parameters of a earthquake.


%\section{Algoritmos Genéticos – O que são?}

%\subsection{Cálculos}

