\chapter{Resultados Obtidos}\label{chapter7}
\section{Simple L-test Fitness Function}
O operador escolhido como parâmetro de {\it crossover} foi o {\it Blend}, e a preferência por ele dentre os demais implementados pelo pacote DEAP foi pela observação empírica de crescimento dos valores de L-test e pelo fato de o operador ser específico para indivíduos formados por números reais, situação encontrada na aplicação. A média do resultado de 10 execuções:  \\
%arrumar dados
%O L-test do melhor indivíduo da população inicial é: 2.57371265858 e o valor mais alto de L-test obtido é: 6328.5844906

\begin{table}[!h]
  \begin{center}
  \begin{tabular}{|l|r|}
    \hline
    Primeira geração (s) & 145.6644\\
    Última geração (s) & 113.4796\\
    Valor do L-test da primeira geração & -0.635949214\\
    Valor do L-test da última geração & -0.009772415\\
    \hline    
  \end{tabular}
  \end{center}
  \caption{Tempo gasto e valor do L-test na média de 10 execuções com Blend.}
  \label{GAParameters--}
\end{table}

Operador de {\it crossover} {\it Two Points} foi utilizado por motivos de comparação. A média do resultado de 10 execuções: \\
%por em tabela
\begin{table}[!h]
  \begin{center}
  \begin{tabular}{|l|r|}
    \hline
    Primeira geração (s) & 129.03775\\
    Última geração (s) & 97.00832\\
    Valor do L-test da primeira geração & -0.635856248\\
    Valor do L-test da última geração & 0.042102222\\
    \hline    
  \end{tabular}
  \end{center}
  \caption{Tempo gasto e valor do L-test na média de 10 execuções com Two Points.}
  \label{GAParameters---}
\end{table}

Tanto o modelo preferido quanto o modelo comparativo foram capazes de evoluírem, obtendo valores finais médios superiores aos valores médios aleatórios iniciais.\\

Surpreendemente, por \citep{janikow1991experimental}, era esperado que representações em ponto flutuante tivessem um desempenho de maior acurácia. Porém, o desempenho do operador específico para números reais, o {\it Blend}, foi menor quando comparado a um operador de uso mais geral, o {\it Two Points}, sendo mais lento e obtendo valores de L-test menores, na média.	\\
	
Todos os dados foram obtidos após a execução do algoritmo em um computador Apple MacBook Pro com processador 2.9 GHz Intel Core i7, memória RAM 8 GB 1600 MHz DDR3 com sistema operacional OS X 10.9.1 (13B42).\\

%Pelo expressivo tempo total da aplicação, durante várias análises, utilizou-se um nú- mero menor de gerações (principalmente quando o objetivo era comparar outros trechos de código que não o trecho de evolução da população).\\


Para melhor visualização e aumentar o poder de comparação, quatro figuras podem ser analisadas a seguir. As Figuras \ref{popXmediaLtest_CXBLEND(1)} e \ref{popXmediaLtest_2POINTS(1)} mostram as médias do valores do L-test para a todas as populações enquanto que as figuras \ref{popXmediaG_CXBLEND(1)} e \ref{popXmediaG_2POINTS(1)} mostram as médias valores do tempo também para todas as populações. Figuras \ref{popXmediaLtest_CXBLEND(1)} e \ref{popXmediaG_CXBLEND(1)} são referentes a execuções com o {\it crossover Blend} e as figuras\ref{popXmediaLtest_2POINTS(1)} e \ref{popXmediaG_2POINTS(1)}, com o {\it crossover Two Points}. \\

%A mutação, FlipBit foi escolhida por estar presente nos exemplos fornecidos pelo pacote e ter sido facilmente entendida e pois dentre os operadores mutação implementados não foram de agrado, e futuramente deverá ser criada uma função específica para a aplicação.\\

Por observação empírica das diversas execuções realizadas o maior tempo gasto é com cálculos de L-test, sendo influenciado principalmente pela quantidade de observações e pelo tamanho escolhido para {\it bins}. Quanto menor for o tamanho escolhido para o {\it bin} maior será a resolução do terreno analisado e, conseqüentemente, mais informações sobre ele teremos e maior será o espaço de busca, aumentando o tempo total gasto.\\

Há um grande aumento do valor do L-test entre a vigésima geração e a quadragésima geração. Isso significa que a função de {\it fitness} resulta em um excessivo valor de {\it overfitting}, um super ajuste a base de dados.\\
%onde fica isso?
% Possivelmente, alterar a função utilizada pela mutação e alterar seu valor de acordo com a população deve ser capaz de minimizar a questão.



%Em um segundo momento, baseado nas observações anteriores, algumas modificações foram realizadas.
%%FIGURAS%%
%1
%\begin{figure}[!htb]
%\centering
%\includegraphics[scale=0.4]{popXmediaLtest_CXBLEND(1)}
%\caption{Média dos valores de L-test de cada população( com {\it crossover Blend.})}
%\label{popXmediaLtest_CXBLEND(1)}
%\end{figure}
%
%%2
%\begin{figure}[!htb]
%\centering
%\includegraphics[scale=0.4]{popXmediaLtest_2POINTS(1)}
%\caption{Média dos valores de L-test de cada população (com {\it crossover Two Points.})}
%\label{popXmediaLtest_2POINTS(1)}
%\end{figure}
%
%%3
%\begin{figure}[!htb]
%\centering
%\includegraphics[scale=0.4]{popXmediaG_CXBLEND(1)}
%\caption{Média tempo gasto (em segundos) por cada população, com {\it crossover Blend.}}
%\label{popXmediaG_CXBLEND(1)}
%\end{figure}
%
%%4
%\begin{figure}[!htb]
%\centering
%\includegraphics[scale=0.4]{popXmediaG_2POINTS(1)}
%\caption{Média do tempo gasto (em segundos) por cada população, com {\it crossover Two Points.}}
%\label{popXmediaG_2POINTS(1)}
%\end{figure}

\section{Time-slice Log Likelihood Fitness Function}

Os resultados da aplicação GA {it Log Likelihood  Fitness Function} serão demonstrados a seguir.\\

Foram feitas comparações entre os resultados dos valores de {\it fitness} dos melhores indivíduos e o desvio padrão da população a que ele pertence. Esses resultados comparados são referentes ao estudo dos grupos de operadores e a técnica de pesos adaptativos. A seguir, além dos resultados obtidos, algumas Figuras utilizadas para comparações serão mostradas.\\

As Figuras \ref{2000-media_melhores-1point} e \ref{2010-media_melhores-1point} demonstram o resultado das 50 execuções, mostrando dados referentes a execuções do conjunto de operadores {\it One Point, Worst} e {\it Shuffle Indexes} para os anos de 2000 (quando há um ganho de desempenho ao utilizar-se a tabela) e de 2010 (quando há uma perda de desempenho). As Figuras mostram o melhor indivíduo e o desvio padrão de sua população. Está claro que existe os resultados não são consistentes e que existe alguma falha na aplicação.\\

Essa falha ficou clara após a mudança do cálculo de fatorial pela tabela em memória. Até onde x = 27 (x se refere ao número da execução), os valores obtidos vieram de cálculos do fatorial, a partir disso, foi utilizada a tabela. Porém, devido a essa falha, não é possível qualificar se algum grupo gerou indivíduos mais aptos. Portanto é essencial corrigi-la para novas execuções sejam geradas e as comparações possam ser refeitas.\\

Foram duas as causas levantadas para essa falha. A primeira, refere-se ao valor limitante para a tabela do fatorial. Uma vez que a tabela possui valores até o fatorial de 100, quaisquer valores acima desse limitante terão seus fatoriais arredondados a 100. Há casos que essa aproximação foi vantajosa e a aplicação respondeu com modelos de melhor desempenho, mas em alguns casos, a aproximação foi desvantajosa.\\

A segunda, refere-se a um erro no código da execução. Após uma análise superficial do código, 
é provável que o mapeamento dos dados coletados esteja propagando algum erro. Esforços futuros serão direcionados para a correção deste erro.\\


%As Figuras \ref{2000-media_melhores-1point} e \ref{2010-media_melhores-1point} exemplificam o ocorrido, mostrando dados referentes a execuções do conjunto de operadores {\it One Point, Worst} e {\it Shuffle Indexes} para os anos de 2000 (quando há um ganho de desempenho ao utilizar-se a tabela) e de 2010 (quando há uma perda de desempenho). Até onde x = 27, os valores obtidos vieram de cálculos do fatorial, a partir disso, foi utilizada a tabela.\\

Ainda assim, a tabela em memória do valores do fatorial resultou em um avanço em termos de performance temporal. Uma vez que a segunda possível causa seja o verdadeiro erro da aplicação, acredita-se que o uso da tabela será capaz de facilitar futuras execuções.\\

Os resultados obtidos pela técnica de pesos adaptativos não foram tão elevadas quanto o esperado. Para efeitos comparativos dois grupos de figuras são mostrados a seguir. As Figuras \ref{2000-media_melhores} e \ref{2000-AUTO-media_melhores} mostram os valores dos melhores indivíduos e o desvio padrão da população para o ano de 2000 e compõem o primeiro grupo, já o segundo grupo é composto pelas figuras \ref{2010-media_melhores} e \ref{2010-AUTO-media_melhores} e segue o mesmo princípio, porém para o ano de 2010. No primeiro grupo, a Figura \ref{2000-media_melhores} refere-se aos dados da técnica em questão, enquanto que a outra Figura refere-se ao mesmo conjunto de operadores utilizados ({\it crossover}, {\it Two Points} e {\it Shuffle Indexes}), mas com pesos fixos. O segundo grupo é descrito da mesma maneira.\\

Não fica claro, pelos dados mostrados, quais são os benefícios que a técnica introduz para a aplicação em questão. Após os mais diversos tipos de análises, análises sobre a influência dos grupos de operadores, seja na própria aplicação GA ou mesmo nas funções da CEC'13 e análises sobre diferentes técnicas para estruturação da abordagem GA, não ficou claro qualquer tipo de variação substancial capaz de direcionar os estudos e definir uma abordagem única para os experimentos. Portanto, deve-se fazer a devida ponderação: será que as variações realmente influem pouco nos resultados da aplicação, ou a aplicação deve passar por refinamentos, que levem a resultados mais consistentes?\\

Para responder essa ponderação devemos analisar as escolhas feitas para a criação de nosso modelo. Pela própria estrutura definida pelo modelo proposto, cada {\it bin} é tratado independentemente dos seus vizinhos, não considerando a influência de sismos próximos. Porém, pela características inerentes aos sismos é claro que poucos são os casos de sismos completamente independentes entre si. Podemos considerar, logo, que a influência da independência influi consideravelmente e é ela a principal causa da mínima variação pré-citada. \\

Por outro lado, as funções CEC'13 são funções unicamente matemáticas e por isso a questão levantada anteriormente não é suficiente para sugerir que a resposta a nossa ponderação deva ser direcionada somente para o segundo ponto da pergunta, ``a aplicação deve passar por refinamentos''. Devemos deixar claro, entretanto, que essa independência entre os {\it bins} deve ser melhor explorada a fim de encontrar uma solução que possibilite considerar a dependência dos sismos. Nesse ponto, percebemos que a complexidade da questão é elevada. \\ 

Apesar da utilização da técnica de pesos adaptativos não ter contribuído para uma performance mais significativa, fomos capazes de observar que os valores médios do melhores indivíduos durante os anos foram muito mais coesos do que quando comparados aos valores similares dos ensaios com pesos fixos. Isso nos leva a crer que utilizar a técnica indiretamente influiu positivamente em nossos resultados. Consequentemente, a idéia de abranger uma área maior de busca e posteriormente especifica-la, mostrou-se, como esperado, bastante adequada.\\

Ainda em relação a técnica anterior, é possível que outras contribuições não tenham sido percebidas e que, de acordo com as devidas mudanças, sejamos capazes de percebê-las. Essas mudanças tanto podem ser algumas das já citadas como também na porcentagem de variação dos pesos dos operadores, que é de 30\% (escolhido arbitrariamente), nos valores desses pesos na situação inicial, entre outras, ou seja, variações na estrutura utilizada juntamente com a técnica.\\


%%%FIGURAS%%%%
%5
%\begin{figure}[!h]
%\centering
%\textbf{Ano 2000}\par\medskip
%\includegraphics[scale=0.5]{2000-media_melhores-1point.pdf}
%%\small
%\caption{Melhor indivíduo e desvio padrão da população(y) em cada execução(x), desempenho médio superior com a tabela do fatorial.}
%\label{2000-media_melhores-1point}
%\end{figure}
%
%%6
%\begin{figure}[!h]
%\centering
%\textbf{Ano 2010}\par\medskip
%\includegraphics[scale=0.5]{2010-media_melhores-1point.pdf}
%\caption{Melhor indivíduo e desvio padrão da população(y) em cada execução(x), desempenho médio inferior com a tabela do fatorial.}
%\label{2010-media_melhores-1point}
%\end{figure}
%
%%7
%\begin{figure}[!h]
%\centering
%\textbf{Ano 2000}\par\medskip
%\includegraphics[scale=0.5]{2000-media_melhores.pdf}
%\caption{Melhor indivíduo e desvio padrão da população(y) em cada execução(x), pesos fixos.}
%\label{2000-media_melhores}
%\end{figure}
%
%%8
%\begin{figure}[!h]
%\centering
%\textbf{Ano 2000}\par\medskip
%\includegraphics[scale=0.5]{2000-AUTO-media_melhores.pdf}
%\caption{Melhor indivíduo e desvio padrão da população(y) em cada execução(x), GA com pesos adaptativos.}
%\label{2000-AUTO-media_melhores}
%\end{figure}
%
%%9
%\begin{figure}[!h]
%\centering
%\textbf{Ano 2010}\par\medskip
%\includegraphics[scale=0.5]{2010-media_melhores.pdf}
%\caption{Melhor indivíduo e desvio padrão da população(y) em cada execução(x), pesos fixos.}
%\label{2010-media_melhores}
%\end{figure}
%
%%10
%\begin{figure}[!h]
%\centering
%\textbf{Ano 2010}\par\medskip
%\includegraphics[scale=0.5]{2010-AUTO-media_melhores.pdf}
%\caption{Melhor indivíduo e desvio padrão da população(y) em cada execução(x), GA com pesos adaptativos.}
%\label{2010-AUTO-media_melhores}
%\end{figure}