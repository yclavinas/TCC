\chapter{Introduction}\label{chapter1}
In this chapter we present a general specification of the problem, its relevance and what are the goals of this study.

\section{Earthquakes}
Earthquakes may cause lots of damages environment and consequently may represent, directly or inderectly, a risk to human lives. They manifest themselves by shaking and sometimes displacement of the ground. They may also cause tsunamis, landslides, volcano activities, etc.\\


There are many examples that show how devastating one large earthquake can be. In 25th of April 2015,there was a strong earthquake in Nepal, with moment magnitude of $M_w$ 7.8 and considered the largest since 1934. It destroyed lots of buildings and
infrastructure, and triggered numerous landslides and rock/boulder falls in the mountain areas~\cite{wilkinson20152015}. Many other aftershock occurrences, which are dependent earthquakes~\cite{van2012seismicity}, happened after it, including two major aftershocks $M$ 6.7 and $M$ 7.3 earthquakes that caused additional that were also very destructive~\cite{wilkinson20152015}.\\

Another example happened in March 2011, Japan. It was a 9.0 $M_w$ earthquake~\cite{simons20112011}, and it is considered the most powerful earthquake to ever hit Japan . It caused tsunami waves that reached more than 39 meters, moved the main island in Japan more than 2 meters east and also changed the Earth axis. It was reported that it caused more than 14 thousand deaths, made more than 244,000 people homeless and provoked a meltdown of the Fukushima Daiichi Nuclear Power Plant complex. Many large aftershocks followed the main event~\cite{mimura2011damage}. Also in 2011 a magnitude $M_w$ 7.1 earthquake hit Van,Turkey and caused lots of deaths and great damages.These are only three very recent examples of large earthquake damages of how dangerous earthquakes can be~\cite{irmak2012source}.\\

Those earthquakes, and many others that hazard the human society, have some common characteristics. They not only are powerful quakes but they happened nearby populated areas, which increase the damaged provoked. To minimize as much as possible future earthquake disaster, a lot can be done. That include to developing goos urban planings, for example to build structures with techniques that can withstand the forces of earthquakes, to create earthquake warning systems, to create more precise civil enginnering codes, and such.\\

To be able to prevent as many casualty as possible, we need the patterns and mechanisms behind the occurrence of earthquakes. We need to know if there is any relationship between the earhquake locations and its time of occurrance, how they are related to each other, et cetera. With this information, it is possible to to create better seismic risk forecast models, indicating which regions show a higher probability of earthquake occurrence at certain periods in time. \\

Until by now, it has been difficult to clearly understand the many different seismic variables (hours, magnitude, local, depth,...) influences the quakes and either exists a mathemathical model capable of supplying detailed and precise information about the relations and ways to estimate them. Therefore, to develop a prediction earthquake model can prove itself very complex.\\

\section{Earthquake Prediction}

Earthquake prediction is a polemic subject. No research has even come close to suggesting that individual large scale earthquakes can be predicted~\cite{ecta14} and many scientists think that earthquake prediction may not be fully impossible but that the resources needed for such a prediction may be out of reach~\cite{eberhard2014multiscale}.\\

%TODO: check the nature1999, why?
In the context of this study, we do not aim to predict any individual earthquake and its major characteristics. Our goal relies on the fact that earthquakes do cluster in time and space. We want to use computer techniques to learn and to generate risk models. There is a lot of value behind the study of earthquake mechanisms, with the goal of generating statistical models of earthquake risk~\cite{Nature1999}.\\

In~\cite{Koza2003}, Koza says that Evolutionary Computation (EC) may find, by try trial and error, based on a great amount of data, better solutions for problems that human beings may not find it easy to solve. EC is a family of subfield of artificial intelligence that aim to extract patterns and to solve problems using a great amount of data, by trial and error.We may also say that without any domain knowledge about the problem to be controlled, the EC learns about it by trial and error~\cite{Michie94machinelearning}.\\

EC techniques constitute a category of heuristic search, they are stochastic algorithms and their search method are based on genetic inheritance and surival of the fittest~\cite{michalewicz1996heuristic}. They are interesting to be used specially in cases that are difficult to understand and the knowledge available is not sufficiently available.\\

Based on these information and on the difficulty to understand how earthquakes behave, we want to explore historical earthquake data using EC. It is expected that it will help to find new ideas about earthquakes, their patterns and their mechanisms behind each earthquake occurrence. For doing so, we need first to outline the forecast problem, then verify the suitability of Evolutionary Computation to the problem of generating earthquake forecast models.\\

Next, we will study ways to improve the generated methods using both EC and other computer techniques and any siesmological knowledge. For this phase we want to propose different representations aiming to refire the algorithm performance and to incorporate seismology methods to refine the models proposed.\\

\section{Document Organization}

This document is organized into 7 more chapters. The next, is about teoretical concepts there are useful for undestanding the development of this study. Then, in the chapter~\ref{chapter3}, we will discuss the current state of art regarding Evolutionary Computation (EC) and earthquake risk	 models. \\

The chapter~\ref{chapter4} is about the methods proposed in the context of this study and a detailed description of its characteristics.\\

The next chapter~\ref{chapter5} is about the earthquake data used, also called catalogue. It is also about a statistical analysis of the data and any relevant decisions made regarding the data itself.\\

The chapter~\ref{chapter6} is all about the experiments. The following chapter~\ref{chapter7} is about the analysis of the experiments and the results observed from it.\\

I the last chapter~\ref{chapter8} brings the conclusing of the study as well as a little discussion of the contribuitions of this work and future works.\\


