%%%%%%%%%%%%%%%%%%%%%%%%%%%%%%%%%%%%%%%%
% Classe do documento
%%%%%%%%%%%%%%%%%%%%%%%%%%%%%%%%%%%%%%%%

% Nós usamos a classe "unb-cic".  Deixe apenas uma das linhas
% abaixo não-comentada, dependendo se você for do bacharelado ou
% da licenciatura.

\documentclass[bacharelado]{unb-cic}
%\documentclass[licenciatura]{unb-cic}



%%%%%%%%%%%%%%%%%%%%%%%%%%%%%%%%%%%%%%%%
% Pacotes importados
%%%%%%%%%%%%%%%%%%%%%%%%%%%%%%%%%%%%%%%%

\usepackage[brazil,american]{babel}
\usepackage[T1]{fontenc}
\usepackage{indentfirst}
\usepackage{natbib}
\usepackage{xcolor,graphicx,url}
\usepackage[utf8]{inputenc}
\usepackage{amsmath}
\usepackage{graphicx}
\usepackage{url}


%%%%%%%%%%%%%%%%%%%%%%%%%%%%%%%%%%%%%%%%
% Cores dos links
%%%%%%%%%%%%%%%%%%%%%%%%%%%%%%%%%%%%%%%%

% Veja o arquivos cores.tex se quiser ver que outras cores estão
% pré-definidas.  Utilizando o comando \hypersetup abaixo nós
% evitamos aquelas caixas vermelhas feias em volta dos links.

\input{cores}
\hypersetup{
  colorlinks=true,
  linkcolor=DarkScarletRed,
  citecolor=DarkScarletRed,
  filecolor=DarkScarletRed,
  urlcolor= DarkScarletRed
}



%%%%%%%%%%%%%%%%%%%%%%%%%%%%%%%%%%%%%%%%
% Informações sobre a monografia
%%%%%%%%%%%%%%%%%%%%%%%%%%%%%%%%%%%%%%%%

\title{Earthquake Risk Induction Models with Evolutionary Computation}

\orientador{\prof \dr Marcelo Ladeira}{CIC/UnB}
\coorientador{\prof \dr Claus de Castro Aranha}{Department of Computer Sciences/University of Tsukuba}
\coordenador{\prof \dr Homero Luiz Piccolo}{CIC/UnB}
\diamesano{20}{dezembro}{2013}

\membrobanca{\prof \dr Claus de Castro Aranha}{Department of Computer Sciences/University of Tsukuba}
\membrobanca{\prof \dr Guilherme Novaes Ramos}{CIC/UnB}

\autor{Yuri Cossich}{Lavinas}
\CDU{004.4}

\palavraschave{algoritmos genéticos, sismos, terremotos, log-likelihood}
\keywords{genetic algorithms, earthquake, log-likelihood}



%%%%%%%%%%%%%%%%%%%%%%%%%%%%%%%%%%%%%%%%
% Texto
%%%%%%%%%%%%%%%%%%%%%%%%%%%%%%%%%%%%%%%%

\begin{document}
  \maketitle
  \pretextual

  \begin{agradecimentos}
  to be done
  \end{agradecimentos}
\selectlanguage{brazil}
  \begin{resumo}
  To be done
  %Este projeto trata do desenvolvimento de um modelo de previsão de sismos utilizando algoritmos genéticos. O objetivo do presente trabalho é construir um modelo empírico para estimar a probabilidade de ocorrência de terremotos no Japão com base em dados de sismos ocorridos anteriormente nessa mesma área geográfica. Os dados utilizados para treinamento foram obtidos da Japan Metereological Agency (JMA) e são referentes a sismos ocorridos no Japão entre os anos de 2000 a 2013. O projeto utiliza o cálculo de Log-Likelihood e o L-test, teste sugerido pela Collaboratory for the Study of Earthquake Predictability (CSEP), como função de {\it fitness} do modelo. O algoritmo genético mostrou uma performance superior a de métodos aleatórios, demonstrando ser um método promissor para a abordagem deste problema.
%   Este projeto trata do desenvolvimento de um modelo de previsão de sismos utilizando algoritmos genéticos. O objetivo do presente trabalho é definir um modelo para estimar a probabilidade de ocorrência de terremotos no Japão com base em dados de sismos ocorridos anteriormente nessa mesma área geográfica. Os dados utilizados para treinamento foram obtidos da {\it Japan Metereological Agency} (JMA) e são referentes a sismos ocorridos no Japão entre os anos de 1997 a 2013, sendo que a base de treino foi limitada entre os anos de 2000 e 2010. O projeto utilizou o L-test, teste sugerido pela {\it Collaboratory for the Study of Earthquake Predictability} (CSEP), como função de {\it fitness} e posteriormente, substituímos o {\it L-test} pelo cálculo de {\it Log-Likelihood}. Os resultados finais foram destinados a entender quais ajustes poderiam ser feitos e quais efeitos que causariam na aplicação. Dentre os ajustes se destacam o testes dos operadores de {\it crossover}, seleção e mutação, e a introdução da técnica de pesos adaptativos. 
  \end{resumo}

  \selectlanguage{american}
  \begin{abstract}	
To understand the mechanisms and patterns of the earthquakes is very important to minimize its consequences. In this context, this projects aims to develop an earthquake prevision risk model using Evolutionary Computation (EC). The main goal is to define a method to estimate the probability of earthquake occurrences in Japan using historical earthquake data of a given geographical region. This work is established in the context of the “Collaboratory for the Study of Earthquake Predictability” (CSEP), which seeks to standardize the studies and tests of earthquake prevision models. The method is based in one application of Genetic Algorithms (GA) and aims to develop statistical methods of analysis of earthquake risk. The risk models generated by this application were analyzed by their log-likelihood values, as suggested by the Regional Earthquake Likelihood Model (RELM). They are compared with real data and the models generated by the application of the Relative Intensity Algorithm (RI). The data used was obtained from the Japan Metereological Angency (JMA) and are related with earthquake activity in Japan between the years of 2000 and 2010.
  \end{abstract}
  \selectlanguage{american}

  \tableofcontents
  \listoffigures
  \listoftables

  \textual
  \chapter{Introduction}
In this chapter we present a general specification of the problem, its relevance and what are the goals of this study.

\section{Earthquakes}
Earthquakes may cause lots of damages environment and consequently may represent, directly or inderectly, a risk to human lives. They manifest themselves by shaking and sometimes displacement of the ground. They may also cause tsunamis, landslides, volcano activities, etc.\\

%TODO: explain aftershock
%TODO: i may need to add references of the quakes
There are many examples that show how devastating one large earthquake can be. In April 2015,there was a magnitude 7.8 $M_w$ earthquake in Nepal. It cause 8000 deaths and triggered two avalanches, one of them in the Mount Everest. Also, many people were made homeless, it destroyed UNESCO World Heritage sites and had many aftershocks, including a magnitude $M_w$ 7.3 quake that caused 200 deaths. Another example happened in March 2011, Japan. It was a 9.0 $M_w$ earthquake, and it is considered the most powerful earthquake to ever hit Japan and the 4th most powerful in the world. It caused tsunami waves that reached more than 40 meters, moved the main island in Japan more than 2 meters east and also changed the Earth axis. It was reported that it caused more than 15000 deaths, made more than 200000 people homeless and provoked a meltdown of the Fukushima Daiichi Nuclear Power Plant complex. (see Figure 1.1 for a picture of the devastation) (May i use this picture?). Also in 2011 a magnitude $M_w$ 7.1 earthquake hit Van,Turkey and caused lots of deaths and great damages.These are only three very recent examples of large earthquake damages of how dangerous earthquakes can be.\\

Those earthquakes, and many others that hazard the human society, have some common characteristics. They not only are powerful quakes but they happened nearby populated areas, which increase the damaged provoked. To minimize as much as possible future earthquake disaster, a lot can be done. That include to developing goos urban planings, for example to build structures with techniques that can withstand the forces of earthquakes, to create earthquake warning systems, to create more precise civil enginnering codes, and such.\\

To be able to prevent as many casualty as possible, we need the patterns and mechanisms behind the occurrence of earthquakes. We need to know if there is any relationship between the earhquake locations and its time of occurrance, how they are related to each other, et cetera. With this information, it is possible to to create better seismic risk forecast models, indicating which regions show a higher probability of earthquake occurrence at certain periods in time. \\

Until by now, it has been difficult to clearly understand the many different seismic variables (hours, magnitude, local, depth,...) influences the quakes and either exists a mathemathical model capable of supplying detailed and precise information about the relations and ways to estimate them. Therefore, to develop a prediction earthquake model can prove itself very complex.\\

\section{Earthquake Prediction}

Earthquake prediction is a polemic subject. No research has even come close to suggesting that individual large scale earthquakes can be predicted~\cite{ecta14} and many scientists think that earthquake prediction may not be fully impossible but that the resources needed for such a prediction may be out of reach~\cite{eberhard2014multiscale}.\\

%TODO: check the nature1999 
In the context of this study, we do not aim to predict any individual earthquake and its major characteristics. Our goal relies on the fact that earthquakes do cluster in time and space. We want to use computer techniques to learn and to generate risk models. There is a lot of value behind the study of earthquake mechanisms, with the goal of generating statistical models of earthquake risk~\cite{Nature1999}.\\

In~\cite{Koza2003}, Koza says that Evolutionary Computation (EC) may find, by try trial and error, based on a great amount of data, better solutions for problems that human beings may not find it easy to solve. EC is a family of subfield of artificial intelligence that aim to extract patterns and to solve problems using a great amount of data, by trial and error.We may also say that without any domain knowl- edge about the problem to be controlled, the EC learns about it by trial and error~\cite{Michie94machinelearning}.\

%TODO": what are black box problems?
EC are based on methaheuristic or stochastic optimization, and are mostly applied for black box problems. They are interesting to be used specially in cases that are difficult to understand and the knowledge available is not sufficiently available.\\

Based on these information and on the difficulty to understand how earthquakes behave, we want to explore historical earthquake data using EC. It is expected that it will help to find new ideas about earthquakes, their patterns and their mechanisms behind each earthquake occurrence. For doing so, we need first to outline the forecast problem, then verify the suitability of Evolutionary Computation to the problem of generating earthquake forecast models.\\

Next, we will study ways to improve the generated methods using both EC and other computer techniques and any siesmological knowledge. For this phase we want to propose different representations aiming to refire the algorithm performance and to incorporate seismology methods to refine the models proposed.\\


%\section{Objetivos}
%Este trabalho visa construir um modelo baseado em dados históricos para estimar ocorrência de sismos utilizando dados reais obtido pela {\it Japan Metereological Agency} (JMA). O modelo gera indivíduos que são possíveis soluções para o problema e os indivíduos de cada população são avaliados pela função de {\it fitness} que utiliza testes sugeridos pela {\it Study of Earthquake Predictability} (CSEP).\\
%
%O objetivo principal do trabalho é mostrar que criar um modelo de previsão de terremotos desenvolvido a partir de Computação Evolutiva é possível e promissor, como também garantir uma qualidade do modelo perante outros modelos de previsão de riscos. %Esperamos atrair mais atenção para futuras pesquisas nessa área.
%%arrumar aqui 
%

  \chapter{The Earthquake Forecasting Problem}~\label{chapter2}
This chapter focus on the teoretical concepts used as base for this study. The main topics are genetic algorithm, the CSEP framework and the siesmologic methods.\\

\section{What are Genetic Algorithms}
The main goal of a Genetic Algorithm (GA) is to find approximated solutions in problems of search and optimization. Based on Koza~\cite{Koza2003} , GA are mechanism of search based on natural selection and genetic. They explore historical data to find optimum search points with some performance increment, as said by Goldberg~\cite{Goldberg:1989:GAS:534133}.\\

\subsection{How does GA work}

A GA uses those mechanisms to generate solutions to optimization and search problems. The first step is to create an initial population of possible solutions. Frequently, the initial population is randomly generated once it is common to ignore the main aspects that influence the algorithm performance.\\

Each possible solution of a population is called an individual. Every individual is a possible solution of a problem. Those individuals have its fitness value estimated by a fitness function. A fitness function should determine how suitable a individual is to a given problem. The most suitable individuals are graded with better values and the not so suitable ones have a lower value.\\

After measuring the population fitness value, some individuals are then selected by a process that takes into account each individual fitness value to influence the next population. The individuals with better values have a higher chance to be selected. The individuals selected take part in the varation process. This process may alter some of the individual characteristics using the crossover and mutation operators.\\

The crossover operator is a operator that is used to vary the characteristics of a group of individuals. For that a number of parents, a group of individuals from the current population, are selected. In most of the cases, the parents are chosen to compose a pair that will exchange information that will take compose the child, a new individual that will belong to the next generation.\\

Another important operator is called the mutation operator. It is a operator with the purpose of avoiding the loss of important information. It works by changing the characteristics of an individual, looking to add new information to the next population.\\

It is common to have a evolutionary operator that allows the fittest individual from the current generation to take part in the next generation. This operator is called Elitism and it is used to assure that the next generation best solution is at least as good as in the current generation.\\


\section{Earthquake Likelihood Model Testing}~\label{testing}

We started studies of the earthquake forecasting problem by determining and selecting ways to build earthquake forecast models, to evaluate and to compare them, as suggested by the Collaboratory for the Study of Earthquake Predictability (CSEP). It is an international partnership to promote rigorous study of the feasibility of earthquake forecasting and predictability~\cite{ecta14}.\\

For that, we gathered some important information about earthquake predicting needed for this study. Most of it is based on the paper {\it Earthquake Likelihood Model Testing} \cite{schorlemmer2007earthquake}. From this paper we gathered information that guided us into how to build, evaluate and compare earthquake forecast models efficiently.\\

A very difficult and yet very common problem when studying earthquake models is how to compare different kinds of models, that are based on different tests protocols. The CSEP proposes a methodology for rigorous scientific testing of these many different models. This group proposed an framework called The CSEP framework. It provides a method to compare earthquakes risk models in an objectively and consistently way~\cite{ecta14}.\\

All forecast models proposed in this study are based in the Collaboratory for the Study of Earthquake Predictability (CSEP) framework. In the CSEP framework, a forecast model uses a gridded rate forecast \cite{zechar2010evaluating}, one common format in the literature. For evaluate and compare these models we used the likelihood based tests. They are the L-test, the N-test and the R-test, as suggested by Regional Earthquake Likelihood Model (RELM)~\cite{schorlemmer2007earthquake}.\\

%TODO: rewrite this
The principle behind each consistency test is the same. One calculates a goodness-of-fit statistic for the forecast and the observed data. One then estimates the distribution of
this statistic assuming that the forecast is the data-generating model (by simulating catalogues that are consistent with the forecast). One then compares the calculated statistic with the estimated distribution; if the calculated statistic falls in lower tail of the estimated distribution, this implies that the observation is inconsistent with the forecast, or that the forecast should be “rejected”. For the CSEP consistency tests used here, the likelihood is the fundamental metric, but this approach would be similar for different statistical measurements~\cite{eberhard2014multiscale}.\\

\subsection{Vector of expectations}~\label{vector}
As stated in section~\ref{testing}, The CSEP framework uses a gridded rate forecast. This gridded forecast may be structured by a vector of earthquake expectations, occurrences probabilities, that are directly related to a vector of real earthquake observations.\\

Based on this structure, it is possible to calculate the Log-likelihood value of a model with the real data observed. It is also possible to use comparison tests based on the calcultation of the Log-likelihood.\\

\subsection{The Log-Likelihood Function}
%All the methods use the log-likelihood value for the fitness function. The fittest individual among all the others, is preserved in the next generation, to make the solution of one generation as good as the its last generation.a gene of the genome representation 
To calculate the Log-likelihood value we need both vectors cited above, in section~\ref{vector}. One of them is the vector of earthquake expectations and the other is the vector of real earthquake observations. On them, each element is considered a bin. \\

Each bin, $b_n$, define the set $\beta$ and $n$ is the size of the set $\beta$:
\begin{equation} 
\beta := {b_1,b_2,...,b_n},n = |\beta|.
\end{equation}
The probability values of the model $j$, expressed by the symbol
$\Lambda$, is made of expectations $\lambda_i^j$ by bin $b_i$. The
vector is define as:
				
\begin{equation}
	\Lambda^j = 
\begin{pmatrix}
    \lambda_1^j, 
    \lambda_2^j, 
    \hdots,
    \lambda_i^j
  \end{pmatrix}
  ;\lambda_i^j := \lambda_i^j(b_i),b_i \in \beta
\end{equation}
		
The vector of earthquake quantity expectations is defined as:
earthquake by time. The $\Omega$ vector is composed by observations
$\omega_i$ per bin $b_i$, as the $\Lambda$ vector:

\begin{equation}
\Omega = 
\begin{pmatrix}
    \omega_1,
    \omega_2,
    \hdots,
    \omega_i
  \end{pmatrix}
  ;\omega_i =\omega_i(b_i),b_i \in \beta
\end{equation}

The calculation of the log-likelihood value for the $\omega_i$
observation with a given expectation $\lambda$ is defined as:


\begin{equation}
	L(\omega_i|\lambda_i^j) = -\lambda_i^j + \omega_i\log\lambda_i^j - \log\omega_i!
\end{equation}

The joint probability is the product of the likelihood of each bin, so
the logarithm $L(\Omega|\Lambda^j)$ is the sum of for
$L(\omega_i|\lambda_i^j)$ every bin $b_i$:

\begin{equation}\label{log-like}
\begin{split}
	L^j = L(\Omega|\Lambda^j) = \sum_{i=1}^{n}L(\omega_i|\lambda_i^j)  \\
	= \sum_{i=1}^{n} -\lambda_i^j + \omega_i\log\lambda_i^j - \log\omega_i!  
\end{split}
\end{equation}

The fitness function is a coded version of the equation
~\ref{log-like}. It uses the probabilities of the bins of each
individual of model for the $\lambda$ values.\\
				
\subsection{Uncertainties in Earthquake Parameters}
It is important to say that the earthquake parameters, as the location, magnitude and focal time, cannot be estimated without uncertainties. Therefore, each parameter uncertainty has to be included in the testing~\cite{schorlemmer2007earthquake}. Moreover, by estimating it, it is possible to judge the reliability and robustness of the forecast testing~\cite{eberhard2014multiscale}. Also, each observation must be treated as independent ones. This is not the case of the aftershocks, once they are directly dependent with another stronger earthquake. \\

\section{Tests for evaluating Models}\label{Tests}
In the paper {\it Earthquake Likelihood Model Testing}~\cite{schorlemmer2007earthquake}, it is proposed some statistical tests that are used in this study, developed by the The
Regional Earthquake Likelihood Models (RELM). They were used to compare
and evaluate the every forecast models. These tests are based on the
log-likelihood score that compares the probability of the model with
the observed events.\\

To evalute the data-consistency of the forecast models we used the
N-Test, the Number Test, and the L-Test, or Likelihood Test. These
tests fall are significance tests. Therfore, assuming a given forecast
model as the null hypothesys, the distribuition of an observable test
is simulated. If the observed test statistic falls into the upper or
lower tail of this distribuition, the forecast is
rejected~\cite{schorlemmer2010first}.\\

To be able to compare the model that passed the N-Test and the L-test,
the R-Test, the hypotheses Comparison Test, is used. It calculates the
relative performance of a model, by comparing the Log-likelihood
values between two forecast models.\\
%TODO: zechar files format
\subsection{L-test - Data-consistency test}
The L(ikelihood)-Test considers that the likelihood value of the model
is consistent with the value obtain with the simulations. The value is
calculated by fowlling the formula, where $\widehat{L}_k$ is the value of the
Log-likelihood of the model {\it j}, in the {\it bin} {\it i} and
$\widetilde{L}$ is the value of the Log-likelihood of the simulation
{\it j} in the {\it bin} {\it q}:


\begin{equation}
\gamma^{j}_{q} = \frac{\left| \left\{ \widehat{L}^j_k | \widehat{L}^j_k \leq \widetilde{L}^j_q, \widehat{L}^j_k \in \widehat{L}^j, \widetilde{L}^j_q \in \widetilde{L}^j  \right\} \right|}  {|\widehat{L}^j|}
\end{equation}

The analysis of the results can be splited into 3 categories, as follows:

\begin{enumerate}
\item Case 1: $\gamma^{j}$ is a low value, or in other words, the
  Log-likelihood of the model is lower then most of the Log-likelihood
  of the simulations. In this case, the model is rejected.
\item Case 2: $\gamma^{j}$ falls near the half of the values obtained
  from the simluations and is consistent with the data.
\item Case 3: $\gamma^{j}$ is high. This means that the Log-likelihood
  of the data da is higher that the Log-likelihood of the model and no
  conclusion can be made what so ever.
\end{enumerate}


It is important to highlight that no model should be reject in case 3,
if based only on the L-Test. In this case the consistency can or
cannot be real, therefore these model should be tested by the N-Test
so that further conclusions can be done.\\

\subsection{Number test or N-Test}
The N(umber)-Test also analises the consistency of the model, but it
compares the number os observations with the number of events of the
simulations. This test is necessary to supply the underpredicting
problem, which may pass unnoticed by the L-Test.\\

This mesure is estimated by the fraction of the total number of
observations by the total number of observations of the model.\\

As the L-test, if the number of events falls near the half of the
values of the distruition, then the model is consistent with the
observation, nor estimating too much events nor too few of them.\\

\subsection{Hypotheses Comparison Test or R-Test}

The Hypotheses Comparison, or the R(atio)-Test, compares two forecast
models against themselves. The log-likelihood is calculted for both
models and then the difference between them is calculated, named the
observed likelihood ratio. This value indicates which one of the model
better fits the observations.\\

The likelihood ratio is calculated for each simulated catalog. If the
fraction of simulated likelihood ratios less than the observed
likelihood ratio is very small, the model is reject.  To make this
test impartial, not given an advantage to any model, this procedure is
applied symmetrically~\cite{schorlemmer2010first}.\\


\subsection{Evaluation}\label{eval}
This section may need to be placed elsewhere.

The evaluation process is made as follow: First, the data-consitency
is tested by the L-Test and the R-test. If the model passes these
tests, meaning that it was not rejected by them, they ares compared
with other forecast models, which were also not reject, with the
R-Test. The model that best fits the R-Test is then chose as the best
model~\cite{schorlemmer2007earthquake}.\\

%Neste trabalho, inicialmente o L-test foi utilizado para calcular a função de {\it fitness} do modelo e posteriormente substituído pelo cálculo do {\it Log-likelihood}. O N-test, que analisa a consistência do modelo ao comparar a quantidade de ocorrências do modelo com os dados reais, não foi utilizado porque a aplicação não calcula novos valores para ocorrências de sismos, logo os dados necessários para os devidos cálculos não estão disponíveis assim como R-test não foi utilizado por comparar dois modelos gerados e não o modelo gerado com os dados reais, como objetiva a aplicação. Nas próximas seções estão explicitados os detalhes sobre esses conceitos.\\
  %\chapter{State of Art}~\label{chapter4}
In this chapter we will briefly about Genetic Algorithms and then discuss some reports of the application of Evolutionary Computation and related method for Earthquake Risk Analysis.\\

\section{What are Genetic Algorithms}
The main goal of a Genetic Algorithm (GA) is to find approximated solutions in problems of search and optimization. Based on Koza~\cite{koza2003genetic}, GA are mechanism of search based on natural selection and genetic. They explore historical data to find optimum search points with some performance increment, as said by Goldberg~\cite{Goldberg:1989:GAS:534133}.\\

\subsection{How does GA work}

A GA uses those mechanisms to generate solutions to optimization and search problems. The first step is to create an initial population of possible solutions. Frequently, the initial population is randomly generated once it is common to ignore the main aspects that influence the algorithm performance.\\

Each possible solution of a population is called an individual. Every individual is a possible solution of a problem. Those individuals have its fitness value estimated by a fitness function. A fitness function should determine how suitable a individual is to a given problem. The most suitable individuals are graded with better values and the not so suitable ones have a lower value.\\

After measuring the population fitness value, some individuals are then selected by a process that takes into account each individual fitness value to influence the next population. The individuals with better values have a higher chance to be selected. The individuals selected take part in the varation process. This process may alter some of the individual characteristics using the crossover and mutation operators.\\

The crossover operator is a operator that is used to vary the characteristics of a group of individuals. For that a number of parents, a group of individuals from the current population, are selected. In most of the cases, the parents are chosen to compose a pair that will exchange information that will take compose the child, a new individual that will belong to the next generation.\\

Another important operator is called the mutation operator. It is a operator with the purpose of avoiding the loss of important information. It works by changing the characteristics of an individual, looking to add new information to the next population.\\

It is common to have a evolutionary operator that allows the fittest individual from the current generation to take part in the next generation. This operator is called Elitism and it is used to assure that the next generation best solution is at least as good as in the current generation.\\

\section{Evolutionary Computation and Earthquake Risk Prevision}
%ANN
The usage of Evolutionary Computation in the field of earthquake risk models is somewhat sporadic.\\

Zhang and Wang~\cite{Zhang2012} used Genetic Algorithms to fine tune an Artificial Neural Network (ANN) and use this system to produce a forecast model. They integrate the global searching from the GA with the local searching ability of the ANN proposed a new model called GA-BP ANN. It optimize the initial weights and thresholds of the ANN and then, it trains the ANN. They compared the new model with its older version and found that the GA-BP ANN can make better network configurations and can improve the efficiency, precision and stability of earthquake risk prediction.\\

Zhou and Zu~\cite{Feiyan2014} make a very similiar work as the one proposed by Zhang and Wang. In this case they proposed a combination of BP ANN with the Levenberg–Marquardt algorithm and their system only forecasts the magnitude parameter of earthquakes. Sadat, in the paper~\cite{sadat2015application}, follows the idea of Zhou and Zu, aiming to predict the magnitude of the earthquakes in North Iran, but in this case, he used ANN and GA.\\

All these three work are based on using the available characteristics of the earthquakes that happen in the area of study to create a risk prediction of earthquake or to propose a magnitude range for future earthquakes. They object to consider each variables influence the most the results so that their methods can achieve higher performance. These works may help us to compare and evaluate our method, or part of it.\\

%Fault Model parameters
Nicknam et al.~\cite{Nicknam2010} simulated some components of a seismogram a station and predicted seismograms for another station. They combined  empirical Green’s function (EGF) with GA. the EGF method is used to synthesize acceleration time histories and the GA approach is developed to optimize the seismological model. They found that this method obtained good agreement with the observed data, but are not sure that results are free from uncertainties.\\

In this paper, they work with more than 30 seismological model parameters. We can use this information for two paths of action. The first, we may investigate if more earthaquake parameters will improve our method and the other path is to analyse how they dealt with some many variables. Then we may consider to do the same and inspect the results.\\  

Following the same idea proposed by the paper comment above in the work done by Kennett and Sambridge~\cite{Kennett1992}. They also used GA and associated teleseisms procedures to determine the Fault Model parameters of an earthquake. By doing so, they demonstrated that non-linear inversion can be achieved for teleseismic problems without any calculation of waves travel times. They used only P-wave data and expect that if more data could be introduced, the method would accomplish better results.\\

%PGA
Some sismological models were developed aiming to estimate parameter values by using Evolutionarry Computation. For example, Evolutionary Computation was used to estimate the peak ground acceleration (PGA) of seismically active areas~\cite{Kermani2009,Cabalar2009,Kerh2010,Kerh2015}. \\

The two works done by Kerh~\cite{Kerh2010, Kerh2015} are basely a combination of ANN and GA to estimate or predict PGA in Taiwan. These work are based on the benefits of mixing both techniques. They state that the usage of a purely ANN method to estimate PGA may fall into a local minimum and that can be avoid by combining ANN with GA, hence GA is a good method to find global optimums. By doing that the new ANN+GA method will achieve more reliable results.\\

They aimed to decide which areas may be considered potentially hazardous areas. Hence they focus on urban areas, these works are important to revalidate building regulations, urban development and such. The earthquake variables that were used in these work are: local magnitude, focal depth, and epicenter distance. Both magnitude and depth are already used in our work, which is not the case of the epicenter distance variable. They also state that PGA is inversely proportional to epicenter distance, so this variable may be useful to our work by adding useful information to predict risk models both direct or indirectly.\\
 
Ramos~\cite{Ramos2011} used Genetic Algorithms to decide the location of sensing stations. In this work Ramos achieved, in general, better results with the GA method when compared with the seismic alert system (SAS) method and a greedy algorithm method. In some cases, the SAS has a better reponse time than the GA. They consider it to be once caused because the SAS only alerts when earthquakes with magnitude bigger than 5.0 degress in the Richter scale occurs, while the GA deals with all the earthquakes.\\

Ramos's work is a important work because it helps the population to avoid bigger disasters caused by earthquakes by incrensing the time response of the Seismic-Sense Stations. It has some similar feature as the one present in this document: it uses GA to prevent earthquake disasters and tries to locate targets in a given area (though the targets of this work is sensing stations and ours works target is the earthquakes location) and it proposes a way to do GA parameter setting to find which combination of values for the GA parameters achieve highier results. It is interesting to state that once a solution places a station in an area that is not possible to be have sensors, this solution suffers some penalities.\\

Saeidian~\cite{saeidian2016evaluation} also based on the same idea of locating sensing stations. His work differs a little when compared with one above because it makes a comparation in performance between the GA and Bees Algorithm to decide which of those techniques would performe better when chosing the location of sensing stations.\\

Huda and Santosa \cite{ijse5762} published a paper in which the goal is to find, via Genetic Algorithm, the speed of the wavesP and S in the mantle and in the earth crust. P waves are indicated as the first fault found  in seismological data and S waves are the changes caused in the phase of a P wave~\cite{ijse5762}. This research aims to obtain a structure of the japanese undergound and geographically focuses in the same region as our work.\\
  \chapter{Computação Evolutiva e Previsão de Sismos}~\label{chapter4}
%TODO: get this info from main.pdf
%TODO: update this information
%A relação Computação Evolutiva e Previsão de Sismos ainda é escassa, pouco explorada. Esse capítulo é dedicado a mostrar o que já foi explorado dessa relação pela comunidade acadêmica. \\

O presente capítulo é dedicado a mostrar o que já foi explorado da relação entre Previsão de Sismos e Computação Evolutiva (EC) pela comunidade acadêmica.% Apesar de ser uma relação pouco estudada, a seguir descreveremos trabalhos correlatos.  \\

Uma das abordagens utilizada é a hibridização entre as técnicas de Computação Evolutiva (EC). Zhang e Wang \cite{Zhang2012} utilizaram Algoritmo Genético (GA) para refinar uma Redes Neurais (ANN) e, a partir dessa aplicação, criar um modelo de previsão. Zhou and Zhu \cite{zhou2014earthquake}, para realizar uma previsão da magnitude de sismos, fizeram uma combinação entre ANN e EC.\\

Muitas das aplicações estimam características dos sismos ou de suas atividades, como por exemplo calcular o {\it Peak Ground Acceleration} (PGA), \cite{pga_Kerh, Kermani2009, Cabalar2009}. PGA é uma medida da aceleração do sismo no solo e pode ser utilizada, por exemplo, para projetar estruturas mais resistentes a abalos sísmicos, tendo importância elevada em áreas próximas ao cinturão sísmico \cite{Cabalar2009}.\\

GA já foi utilizado também para decidir a localização, baseado em atividades sísmicas, de estações de sensoriamento no México \cite{Ramos2011}. Ele foi utilizado como uma ferramenta de projeto para construir uma rede de estações em diferentes regiões do México, objetivando formar uma rede com estações em locais ótimos, afim de alertar a população o mais rápido possível para evitar maiores desastres.\\

Já Nicknam \cite{Nicknam2010} e Kennett e Sambrigde \cite{Kennett1992} utilizaram EC para determinar parâmetros para modelos de falhas (como epicentro, localização, profundidade, etc.) de um dado sismo.\\%detalhar mais coisas!!!!!

Huda e Santosa \cite{ijse5762} recentemente publicaram um artigo em que buscam determinar, com algoritmos genéticos, a velocidade das ondas P e S no manto e na crosta terrestres. Ondas P são indicadas como a primeira falha encontrada em dados sismológicos e ondas S são as mudanças causadas na fase das ondas P \cite{ijse5762}. Essa pesquisa busca obter a estrutura do subsolo japonês e, geograficamente, possui o mesmo foco que a presente pesquisa.


In this section we will briefly discuss some reports of the
application of Evolutionary Computation and related method for
Earthquake Risk Analysis.

The usage of Evolutionary Computation in the field of earthquake risk
models is somewhat sporadic. Zhang and Wang~\cite{Zhang2012} used
Genetic Algorithms to fine tune an Artificial Neural Network (ANN) and
use this system to produce a forecast model. Zhou and
Zu~\cite{Feiyan2014} also proposed a combination of ANN and EC, but
their system only forecasts the magnitude parameter of
earthquakes. Sadat, in the paper~\cite{sadat2015application}, follows
the idea oF Zhou and Zu, aiming to predict the magnitude of the
earthquakes in North Iran, but in this case, he used ANN and GA.

Some sismological models were developed aiming to estimate parameter
values by using Evolutionarry Computation. For example, Evolutionary
Computation was used to estimate the peak ground acceleration of
seismically active
areas~\cite{Kermani2009,Cabalar2009,Kerh2010,Kerh2015}. Ramos~\cite{Ramos2011}
used Genetic Algorithms to decide the location of sensing stations and
Saeidian~\cite{saeidian2016evaluation} made a comparation in
performance between the GA and Bees Algorithm to decide which of those
techniques would performe better when chosing the location of sensing
stations. Nicknam et al.~\cite{Nicknam2010} and Kennett and
Sambridge~\cite{Kennett1992} used evolutionary computation to
determine the Fault Model parameters of a earthquake.


%\section{Algoritmos Genéticos – O que são?}

%\subsection{Cálculos}


  \chapter{Análise dos Dados}\label{chapter5}
%TODO: statistical analysis, filtres(?). char, main/afeter shocks, histograms
%pesquisar alguma coisa sobre dados e GA só pra formular uma intro
\section{Dados de sismos}
O foco dessa pesquisa é estudar padrões existentes nas ocorrências de sismos. Para isso é essencial que tenhamos acesso a dados confiáveis, seguros e ricos em detalhes. Pela página da {\it Japan Metereological Agency} fomos capazes de acessar dados fiéis aos nossos interesses. Os dados obtidos são compostos por longitude, latitude, data e horário da ocorrência, magnitude e profundidade. \\
%- JMA\footnote[2]{\url {http://www.jma.go.jp/jma/index.html}}

Para a base de treino, foram separados os sismos considerados mais uniformes, formando o grupo cuja latitude e longitude na superfície representam abalos em áreas terrestres (terremotos), pouco profundos (acima de 20 km de profundidade) e com magnitude acima de 2.5 grau na escala Richter, durante os anos de 2000 a 2013.\\

Para selecionar terremotos, foi preciso extrapolar as informações contidas nos dados e buscar uma forma de, a partir dos dados disponíveis, deduzir quais sismos serão considerados terremotos e quais serão considerados maremotos. Para isso foi utilizamos {\it The Google Elevation API}\footnote[3]{\url {https://developers.google.com/maps/documentation/elevation/?hl=pt-es}}, uma {\it Application Programming Interface} (API) que tem como objetivo fornecer informações sobre todos os locais sobre a superfície terrestre e oceânica. \\

A API foi utilizada por oferecer uma interface HTTP para consulta de dados de elevação territorial, que recebe via url os parâmetros latitude e longitude e retorna, em formato {\it JavaScript Object Notation} (JSON), dados como a altitude relativa à coordenadas fornecidas, dado este necessário para nossos estudos. Os abalos cuja coordenadas na superfície terrestre tivessem altitude relativa ao nível do mar acima de 0.0 foram selecionados.\\

Ao analisar a nova base de treino percebe-se um elevado aumento na quantidade de terremotos em 2011, ano que o terremoto de Tohoku, magnitude 9 na escala Ritcher, ocorreu, Figura \ref{ocorrenciasAno}. Esse terremoto causou aumento desproporcional de sismos em todo o Japão, fato que levou a decisão de limitar a base de treino até o ano de 2010.\\

%pq???
Ainda em relação a base de treino, ela foi alterada para gerar fatias anuais da base de treino anterior, baseadas nos dados cronológicos disponíveis nos dados da JMA. As fatias são definidas da seguinte forma: se a base refere-se a sismos ocorridos em um espaço de 10 anos, a base será divida por 10, gerando fatias anuais (por exemplo, de 2004 até 2005). Dessa forma, espera-se minimizar o {\it overfitting} (super ajustes) a base de dados.\\

{\it Overfitting} é o uso de modelos ou procedimentos que violam parcimônia, isto é, que incluem mais termos que o necessário ou usam abordagens mais complicadas do que necessárias \cite{hawkins2004problem}. Visar minimiza-lo, na aplicação GA, é importante para que tenhamos boas generalizações.\\

Outra limitação imposta a base de dados foi em relação a área analisada. Como buscamos entender os padrões dos sismos, escolhemos quatro regiões do Japão para focar o experimento, Kanto, Kansai, Touhoku e East Japan. A Figura \ref{alljapan} propõe uma visualização dessas regiões no Japão. A seguir descreveremos as quatro regiões.\\

\paragraph{Kanto} Kanto é a região ao redor de Tóquio. É uma área com elevada atividade sísmica durante o período estudado. Essa região foi definida com coordenadas começando em 34.8 Norte, 138.8 Oeste, com  2025 {\it bins}. Cada {\it bin} corresponde a uma área de 25km2.\\

\paragraph{Kansai} Kansai é a região que inclui cidades Kyoto, Osaka e outras. Essa área, ao contrário da região de Kanto, possui uma baixa atividade sísmica. Essa região foi definida com coordenadas começando em 34 Norte, 134.5 Oeste, com  1600 {\it bins}. Cada {\it bin} corresponde a uma área de 25km2.\\

\paragraph{Touhoku} Touhoku é a região definida como a região ao norte da ilha principal japonesa. Ela possui alguns {\it clusters} de atividade sísmicas durante o período estudado. Essa região foi definida com coordenadas começando em 37.8 Norte, 139.8 Oeste, com  800 {\it bins}. Cada {\it bin} corresponde a uma área de 100km2. \\

\paragraph{Leste do Japão} É a região corresponde a costa leste do Japão. Ela se diferencia da área anterior por incluir tanto áreas terrestres como não-terrestres. Foi nessa área que o sismo M9 de 2011 aconteceu. Essa região foi definida com coordenadas começando em 37 Norte, 140 Oeste, com  1600 {\it bins}. Cada {\it bin} corresponde a uma área de 100km2. \\

%\begin{figure}[!htb]
%\centering
%\includegraphics[scale=0.9]{ocorrenciasAno}
%\caption{Quantidade de sismos pelos anos.}
%\label{ocorrenciasAno}
%\end{figure}
%
%\begin{figure}[!htb]
%\centering
%\includegraphics[scale=0.5]{alljapan.png}
%\caption{Áreas priorizadas e suas localização.}
%\label{alljapan}
%\end{figure}



%A última, se refere diretamente a nova base de treino. Ao analizar

1'
  \chapter{Metodologia Proposta}\label{chapter6}
\section{Estrutura da solução}
Antes de começar o desenvolvimento do código do algoritmo genético foi necessário criar estruturas, funções e cálculos, assim como estabelecer quais seriam os dados utilizados, como obtê-los e depois como definir qual grupo dentre os dados seriam utilizados para treino.\\

%arrumar aqui por causa dos dados
A primeira estrutura definida foi relacionada aos {\it bins}, $b \in \beta$. Foi necessário definir qual seria o tamanho ideal para cada {\it bin}, pois, como foi dito no capítulo anterior, a resolução de uma previsão é inversamente proporcional ao tamanho do {\it bin}. Foi decidido que cada {\it bin} teria um valor escolhido para representar a variação geográfica de 5 quilômetros, 0.1 graus, utilizado para a construção de cada {\it bin}.\\

%arrumar aqui por causa dos dados
%Em seguida, era necessário estruturar o vetor de expectativas, $\lambda_i$, baseado nos eventos reais, $\omega_i$. Para isso, primeiramente era fundamental possuir dados confiáveis das observações de sismos ocorridos no Japão, com seus parâmetros disponíveis. Esses dados foram obtidos na página da {\it Japan Metereological Agency} - JMA\footnote[2]{\url {http://www.jma.go.jp/jma/index.html}}. Os dados obtidos eram compostos por longitude, latitude, data e horário da ocorrência, magnitude e profundidade. Para os dados de treino, foram separados os sismos 
%considerados mais homogêneos, formando o grupo cuja latitude e longitude na superfície representam abalos em áreas terrestres (terremotos), pouco profundos (acima de 20 km de profundidade) e com magnitude acima de 2.5 grau na escala Richter, durante os anos de 1997 a 2013. Ao separar os dados em grupos mais similares, espera-se que o algoritmo genético possa aprender características semelhantes que possam existir nos dados.\\
%inserir: The Elevation API provides elevation data for all locations on the surface of the earth, including depth locations on the ocean floor (which return negative values). In those cases where Google does not possess exact elevation measurements at the precise location you request, the service will interpolate and return an averaged value using the four nearest locations.

%{\it The Google Elevation API}\footnote[3]{\url {https://developers.google.com/maps/documentation/elevation/?hl=pt-es}} tem como objetivo fornecer informações sobre todos os locais sobre a superfície terrestre e oceânica. Ela foi utilizada por oferecer uma interface HTTP para consulta de dados de elevação territorial, que recebe via url os parâmetros latitude e longitude e retorna, em formato {\it JavaScript Object Notation} (JSON), o resultado da pesquisa.\\

%Como a informação sobre o local do evento (sismos que caracterizavam terremotos ou não) não estava disponível no dados colhidos da JMA, a {\it Application Programming Interface} (API) acima descrita foi utilizada para calcular a altitude relativa ao nível do mar baseada na latitude e longitude do evento. Os abalos cuja coordenadas na superfície terrestre tivessem altitude relativa ao nível do mar acima de 0.0 foram selecionados.\\

%arrumar o próximo parágrafo
%sensor (required) specifies whether the application requesting elevation data is using a sensor (such as a GPS device) to determine the user's location. Accepts true or false.
%locations (required) defines the location(s) on the earth from which to return elevation data.
%Depois de particionar o resultado da pesquisa, o campo do vetor, {\it results}, na posição {\it elevation} mostra a altitude relativa ao mar. Em seguida, seguem dois exemplos compostos pela url utilizada para a pesquisa e o resultado, em JSON, encontrado. \\

Em seguida, era necessário estruturar o vetor de expectativas, $\lambda_i$, baseado nos eventos reais, $\omega_i$. Após obter os vetores de expectativas, a função do {\it log-likelihood} e a de L-test foram desenvolvidos como especificado na Seção 2.2.1. Para um maior entendimento do seu comportamento, o valor do {\it Log-likelihood} foi calculado tanto para os eventos quanto para um vetor de expectativas e observações pseudo-aleatório, e uma comparação entre eles foi feita. E para o entendimento do L-test, esses valores anteriores foram utilizados como os parâmetros da função.

\section{Definição do Algoritmo Genético - Simple L-test Fitness Function}
Inicialmente, para acelerar a prototipação, foi decidido que o Algoritmo Genético seria implementado utilizando {\it Distributed Evolutionary Algorithms in Python} - DEAP \cite{DeRainville:2012:DPF:2330784.2330799}, um {\it framework} para prototipação rápida que almeja por deixar estruturas de dados e algoritmos transparentes \cite{DeRainville:2012:DPF:2330784.2330799}. Para começar o aprendizado de implementações foram utilizados os exemplos de algoritmo genético disponíveis no sub-diretório do framework como base de aprendizado.\\

Foi priorizado o desenvolvimento de um aplicação simples, para que fosse possível analisar o comportamento da função de {\it fitness}. Por esse motivo, por enquanto, o foco das atividades não foi direcionado para os parâmetros da evolução. Por exemplo, {\it crossover} e mutação ainda possuem valores fixos que não consideram a características da população ou dos indivíduos, porém são capazes de garantir uma evolução da população, não prejudicando o objetivo inicial da aplicação.\\

Para a função de {\it fitness} foi utilizado o L-test, comparando o {\it likelihood} dos eventos com o {\it likelihood} dos indivíduos que compõe a população. O indivíduo mais apto, aquele com maior valor do L-test, era mantido na próxima geração, garantindo que a solução de uma geração seja  igual ou maior que a da geração anterior.\\

Janikow e Michalewicz, em \cite{janikow1991experimental}, indicam que utilizar uma representação em ponto flutuante deve ter maior precisão, ser mais consistente, possuir acurácia em performance maior quando comparada a uma representação por bits. Portanto, a escolha de utilizar indivíduos definidos como vetores de números reais, iniciados pseudo-aleatoriamente, pode ser explicada tanto pelos motivos acima descritos tanto pela definição do próprio problema (calcular um modelo de probabilidades).\\

Uma vez que não foi investido muito esforço na análise dos valores escolhidos, o tamanho da população inicial (assim como os valores dos parâmetros) foi definido por tentativa e erros, até que um tempo de convergência aceitável fosse encontrado.\\


\subsection{Crossover}

Os resultados obtidos a partir de execuções do {\it Simple L-test Fitness Function} utilizaram dois operadores de {\it crossover}, o {\it Blend} e o {\it Two Points}, que serão descritos a seguir.\\


%O operarador de {\it crossover} utilizado foi o cxBlend, implementado pelo pacote DEAP,

O {\it crossover Blend}, (BLX-$\alpha$) é um operador para representações em ponto flutuante. Como foi definido em \cite{takahashi2001crossover}, a geração futura é gerada da seguinte forma:

\begin{enumerate}
  \item Escolha dois pais, $x^1$ e $x^2$, aleatoriamente.
  \item Um valor para cada elemento do filho $x^c_i$ da próxima geração é escolhido aleatoriamente do intervalo $[X^1_i, X^2_i]$ da seguinte distribuição:
\begin{center}
	\begin{equation}
	\begin{split}
		X^1_i = min(x^1_i,x^2_i) - \alpha d_i		\\
		X^2_i = max(x^1_i,x^2_i) + \alpha d_i 		\\
d_i = |x^1_i,x^2_i| \\
	\end{split}
	\end{equation}
\end{center}
onde $x^1_i$ e $x^2_i$ são o i-ésimo elementos de $x^1_i$ e $x^2_i$, respectivamente, e $\alpha$ é um parâmetro positivo.
\end{enumerate}

Herrera, Lozano e Verdegay \citep{herrera1998tackling} afirmam que com valor de $\alpha$ = 0.5 a relação entre convergência ({\it exploitation}) e divergência ({\it exploration}) alcança equilíbrio. A escolha do valor de$\alpha$ da aplicação GA é de 0.5, baseada nesta afirmação e na relação de equilíbrio.\\

O {\it crossover Two Points} foi utilizado para efeito de comparação entre o operador para números reais e operadores tradicionais. Definido por Goldeberg em \cite{Goldberg:1989:GAS:534133}, pode seguir dois passos. Esse {\it crossover} é uma instância do {\it n-point crossover}, que por sua vez é uma generalização do {\it simple crossover}, \cite{herrera1998tackling}.\\

Primeiro, dois pais são escolhidos para a reprodução aleatoriamente. Depois, cada pai é dividido em duas partes em uma posição {\it k} selecionada uniformemente e aleatoriamente entre 1 e o tamanho {\it l} dos indivíduos menos 1, $[1, {\it l} -1]$. O filho é, então, gerado ao trocar os dados das posições ${\it k} + 1$ até o fim do indivíduo.\\

Em \cite{herrera1998tackling}, os autores citam o trabalho de Schaffer, Caruana, Eshelman \cite{schaffer1989study}, para definir o {\it n-point crossover}. O {\it crossover} em questão, segue os mesmos princípios que o {\it simple crossover}, porém escolhe {\it n} posições aleatoriamente e esses segmentos criados que são trocados para gerar o filho. No caso do {\it Two Points}, ${\it n} = 2$.

\subsection{Mutação}

O operador de mutação escolhido foi o {\it FlipBit} que é desenvolvido para implementações binárias. Ele inverte o valor de um atributo escolhido aleatoriamente da entrada do indivíduo e usa uma probabilidade para decidir se um atributo sofrerá a mutação, novamente definido em \citep{Goldberg:1989:GAS:534133}.


\subsection{Seleção}

Para seleção foi utilizado Torneio Simples e Elitismo. O Torneio Simples foi utilizado para selecionar, baseado nos valores de {\it fitness} obtidos, os indivíduos para reprodução. O Elitismo foi empregado a fim de ter-se o melhor membro da população presente na geração seguinte, garantindo que a geração atual seja igual ou superior a anterior, quando analisados valores de {\it fitness}.\\

Os parâmetros utilizados pela aplicação {\it Simple L-test} estão descritos na tabela \ref{GAParameters}.\\

\begin{table}[!h]
  \begin{center}
  \begin{tabular}{|l|r|}
    \hline
    Tamanho da população & 500\\
    Número de gerações & 100\\
    cxBlend $\alpha$ & 0.5\\
    Mutação FlipBit & 0.2\\
    Mutação do atributo (indpb) & 0.05 \\
    Tamanho do torneio & 3\\
    \hline    
  \end{tabular}
  \end{center}
  \caption{Parâmetros utilizados.}
  \label{GAParameters}
\end{table}

\section{Definição do Algoritmo Genético - Time-slice Log-Likelihood Fitness Function}
Após análises realizadas a partir dos resultados obtidos pela metodologia anterior que demonstram a ocorrência de {\it overfitting} (situação descrita no Capítulo 6), algumas alterações foram realizadas.\\

%arrumar aqui por causa dos dados
A primeira mudança foi em relação a função de {\it fitness}. Ela foi modificada, passando-se a utilizar o valor do {\it Log-Likelihood}. Já a segunda mudança, foi em relação a base de treino. Ela foi alterada para gerar fatias anuais da base de treino anterior, baseadas nos dados cronológicos disponíveis nos dados da JMA. As fatias são definidas da seguinte forma: Se a base refere-se a sismos ocorridos em um espaço de 10 anos, a base será divida por 10, gerando fatias anuais (por exemplo, de 2004 até 2005). A última foi em relação a área analisada. Como buscamos entender os padrões dos sismos, escolhemos uma região do Japão para focar o experimento, a região de Kanto. Essa região foi escolhida por conter uma grande quantidade de dados terrestres.\\


%Novamente, como feito anteriormente, a primeira estrutura definida foi relacionada aos {\it bins}.  Foi decidido que cada {\it bin} teria um valor escolhido para representar cada uma da regiões geográficas já definidas, Kanto, Tohoku, Leste do Japão e Kansai. Cada {\it bin} é representado por uma variável do tipo {\it double} que em Python tem precisão IEEE 754 (utiliza 64 bits) e contém a expectativa de ocorrência de terremotos na área por ele delimitada. Para converter os valores de expectativas para um valor inteiro que representasse a quantidade de ocorrências para cada {\it bin}, utilizamos um algoritmo modificado de extração desviada de Poisson\citep{NumericalRecipes} (Capítulo 7.3.12).\\



\subsection{Operadores Estudados}
A escolha dos operadores do {\it Time-slice Log-Likelihood Fitness Function} será feita diferentemente do ocorrido com o {\it Simple L-test Fitness Function}. Para o segundo, as escolhas tiveram de caráter arbitrário e, em contrapartida, as escolhas para o primeiro serão feitas comparações entre o desempenho da aplicação para cada combinação de operadores. \\

A fim de obter resultados consistentes e gerais, foram feitas 50 rodadas de execuções de cada combinação de operadores visando analisar valores médios. Dessa forma, pontos dispersos que não representam o comportamento da aplicação terão pequena ou nenhuma influência nas análises finais. 

\subsubsection{Descrição dos Operadores do pacote DEAP}
%arrumar aqui, descrevendo paramentros de entrada e saida, essas coisas todas dos operadores
%falar de como foi feito mais os testes do cec
Uma vez que operadores específicos ainda não foram definidos, todos os operadores disponíveis no pacote DEAP possíveis de serem testados serão explicados a seguir (com exceção para o {\it Two Points}, o {\it Blend}, {\it Flip Bit} e o {\it Tournament} (já descritos anteriormente). Todas as descrições a seguir foram adaptadas da página do DEAP\footnote[4]{\url {http://deap.gel.ulaval.ca/doc/default/api/tools.html}}.\\

Os parâmetros utilizados pela aplicação {\it Time-slice Log-Likelihood} estão descritos na tabela \ref{GAParameters-2}.\\

\begin{table}[!h]
  \begin{center}
  \begin{tabular}{|l|r|}
    \hline
    Tamanho da população & 500\\
    Número de gerações & 100\\
    Operador de {\it crossover} & 0.9\\
    Operador de mutação & 0.1\\
    \hline    
  \end{tabular}
  \end{center}
  \caption{Parâmetros utilizados.}
  \label{GAParameters-2}
\end{table}

\paragraph{Crossover}
Os operadores de {\it crossover} testados foram: {\it One Point}, {\it Uniform}, {\it Two Points}, {\it Partially Matched}, {\it Ordered} e {\it Simulated Binary}.\\

\subparagraph{One Point}
Executa o {\it crossover} de um ponto. Os dois indivíduos de entrada (pais) são modificados e os indivíduos resultantes (filhos) tem o mesmo tamanho dos pais, respectivamente.\\

\begin{table}[!h]
  \begin{center}
  \begin{tabular}{|l|r|}
    \hline
    Parâmetro (1) & O primeiro indivíduo participante da operação \\
    Parâmetro (2) & O segundo indivíduo participante da operação \\
    Retorno & 2 indivíduos modificados\\
    \hline    
  \end{tabular}
  \end{center}
  \caption{Parâmetros e retorno do {\it One Point}}
  \label{One Point}
\end{table}

\subparagraph{Uniform}
Executa um {\it crossover} uniforme que modifica os dois pais. Os atributos são trocados de acordo com uma probabilidade, usualmente com valor 0.5.\\

\begin{table}[!h]
  \begin{center}
  \begin{tabular}{|l|r|}
    \hline
    Parâmetro (1) & O primeiro indivíduo participante da operação \\
    Parâmetro (2) & O segundo indivíduo participante da operação \\
    Parâmetro (3) & Probabilidade independente para troca de cada atributo\\
    Retorno & 2 indivíduos modificados\\
    \hline    
  \end{tabular}
  \end{center}
  \caption{Parâmetros e retorno do {\it Uniform}}
  \label{Uniform}
\end{table}

\subparagraph{Partially Matched}
Executa, nos pais, um {\it crossover} parcialmente correspondido. Os indivíduos gerados são criados pela correspondência de pares de índices dos pais em um dado intervalo e pela troca dos valores de seus índices.\\

\begin{table}[!h]
  \begin{center}
  \begin{tabular}{|l|r|}
    \hline
    Parâmetro (1) & O primeiro indivíduo participante da operação \\
    Parâmetro (2) & O segundo indivíduo participante da operação \\
%    Parâmetro (3) & Probabilidade independente para troca de cada atributo\\
    Retorno & 2 indivíduos modificados\\
    \hline    
  \end{tabular}
  \end{center}
  \caption{Parâmetros e retorno do {\it Partially Matched}}
  \label{Partially Matched}
\end{table}

\subparagraph{Uniform and Partially Matched}
Executa uma combinação entre o {\it Uniform} e o {\it Partially Matched}. Segue o mesmo comportamento desse último, porém o pareamento dos índices é feito aleatoriamente a partir de uma probabilidade, assim como o {\it Uniform}.\\

\begin{table}[!h]
  \begin{center}
  \begin{tabular}{|l|r|}
    \hline
    Parâmetro (1) & O primeiro indivíduo participante da operação \\
    Parâmetro (2) & O segundo indivíduo participante da operação \\
%    Parâmetro (3) & Probabilidade independente para troca de cada atributo\\
    Retorno & 2 indivíduos modificados\\
    \hline    
  \end{tabular}
  \end{center}
  \caption{Parâmetros e retorno do {\it Uniform and Partialy Matched}}
  \label{Uniform and Partially Matched}
\end{table}

\subparagraph{Ordered}
Gera "buracos" nos indivíduos de entrada. São criados quando um atributo de um indivíduo está entre 2 pontos de outro indivíduo. O elemento é rotacionado de tal forma que todos os elementos entre os pontos de {\it crossover} são preenchidos com os elementos removidos, em ordem.\\

\begin{table}[!h]
  \begin{center}
  \begin{tabular}{|l|r|}
    \hline
    Parâmetro (1) & O primeiro indivíduo participante da operação \\
    Parâmetro (2) & O segundo indivíduo participante da operação \\
%    Parâmetro (3) & Probabilidade independente para troca de cada atributo\\
    Retorno & 2 indivíduos modificados\\
    \hline    
  \end{tabular}
  \end{center}
  \caption{Parâmetros e retorno do {\it Ordered}}
  \c{Ordered}
\end{table}

\subparagraph{Simulated Binary}
Faz um {\it crossover} por simulação binária que modifica os indivíduos de entrada. Espera uma seqüência de indivíduos em ponto flutuante. Além de dois indivíduos, recebe também um valor para $\beta$ (Um $\beta$ alto produz filhos parecidos com os pais e um $\beta$ baixo produz filhos mais diferentes).\\


\paragraph{Mutação}
Os operadores de mutação testados foram: {\it Shuffle Indexes} e {\it Uniform Integer}.\\
\subparagraph{Shuffle Indexes}
Embaralha os atributos do indivíduo de entrada. Geralmente aplicada em vetor de índices.\\

\begin{table}[!h]
  \begin{center}
  \begin{tabular}{|l|r|}
    \hline
    Parâmetro (1) & O primeiro indivíduo participante da operação \\
%    Parâmetro (2) & O segundo indivíduo participante da operação \\
    Parâmetro (2) & Prob. independente para troca de cada atributo para outra posição\\
    Retorno & 1 indivíduo modificado\\
    \hline    
  \end{tabular}
  \end{center}
  \caption{Parâmetros e retorno do {\it Shuffle Indexes}}
  \label{Shuffle Indexes}
\end{table}

\subparagraph{Uniform Integer}
Aplica mutação no indivíduo ao substituir alguns de seus atributos por um inteiro uniformemente retirado de um intervalo definido.\\

\begin{table}[!h]
  \begin{center}
  \begin{tabular}{|l|r|}
    \hline
    Parâmetro (1) & O primeiro indivíduo participante da operação \\
%    Parâmetro (2) & O segundo indivíduo participante da operação \\
    Parâmetro (2) & Prob. independente para troca de cada atributo para outra posição\\
    Retorno & 1 indivíduo modificado\\
    \hline    
  \end{tabular}
  \end{center}
  \caption{Parâmetros e retorno do {\it Uniform Integer}}
  \label{Uniform Integer}
\end{table}

\paragraph{Seleção}
Os operadores de seleção testados foram: {\it Roulette} e {\it Random}, {\it Best} e {\it Worst}.\\

\subparagraph{Roulette}
Seleciona indivíduos a partir de giros da roleta. A seleção é feita levando-se em conta espaços que representam a aptidão relativa de cada indivíduo.\\

\begin{table}[!h]
  \begin{center}
  \begin{tabular}{|l|r|}
    \hline
    Parâmetro (1) & 1 lista de indivíduos a serem selecionados \\
    Parâmetro (2) & K, número de indivíduos a selecionar \\
%    Parâmetro (3) & Prob. independente para troca de cada atributo para outra posição\\
    Retorno & 1 lista dos indivíduos selecionados\\
    \hline    
  \end{tabular}
  \end{center}
  \caption{Parâmetros e retorno do {\it Roulette}}
  \label{Roulette}
\end{table}

\subparagraph{Random}
Seleciona indivíduos aleatoriamente da entrada, com substituição. \\

\begin{table}[!h]
  \begin{center}
  \begin{tabular}{|l|r|}
    \hline
    Parâmetro (1) & 1 lista de indivíduos a serem selecionados \\
    Parâmetro (2) & K, número de indivíduos a selecionar \\
%    Parâmetro (3) & Prob. independente para troca de cada atributo para outra posição\\
    Retorno & 1 lista dos indivíduos selecionados\\
    \hline    
  \end{tabular}
  \end{center}
  \caption{Parâmetros e retorno do {\it Random}}
  \label{Random}
\end{table}

\subparagraph{Best}
Seleciona os melhores indivíduos.\\

\begin{table}[!h]
  \begin{center}
  \begin{tabular}{|l|r|}
    \hline
    Parâmetro (1) & 1 lista de indivíduos a serem selecionados \\
    Parâmetro (2) & K, número de indivíduos a selecionar \\
%    Parâmetro (3) & Prob. independente para troca de cada atributo para outra posição\\
    Retorno & 1 lista dos indivíduos selecionados\\
    \hline    
  \end{tabular}
  \end{center}
  \caption{Parâmetros e retorno do {\it Best}}
  \label{Best}
\end{table}

\subparagraph{Worst}
Seleciona os piores indivíduos.\\

\begin{table}[!h]
  \begin{center}
  \begin{tabular}{|l|r|}
    \hline
    Parâmetro (1) & 1 lista de indivíduos a serem selecionados \\
    Parâmetro (2) & K, número de indivíduos a selecionar \\
%    Parâmetro (3) & Prob. independente para troca de cada atributo para outra posição\\
    Retorno & 1 lista dos indivíduos selecionados\\
    \hline    
  \end{tabular}
  \end{center}
  \caption{Parâmetros e retorno do {\it Worst}}
  \label{Worst}
\end{table}

\clearpage

\section{Estudo dos Operadores - Funções CEC'13}
Com objetivo único de aprimorar os conhecimentos sobres os operadores do DEAP e entender melhor o comportamento de Algoritimos Genéticos em problemas de otimização, as primeiras análises foram feitos sobre a suíte de testes do CEC'13 - Congress on Evolutionary Computation - que inclui 28 funções de referência. Dentre as 28 funções, oito representavam um grupo interessante para nossos objetivos. Todas as oitos são funções compostas que combinam as propriedades das sub-funções além de terem comportamento contínuo próximo ao ótimos locais e global. Para maiores informações acerca das funções, recomendamos a leitura do artigo \citep{liang2013problem}.\\

O CEC, além de fornecer todas as informações necessárias para o entendimento das funções e suas características, ainda disponibiliza códigos da suíte em C, Java e Matlab.\\% para {\it download}\footnote[5]{\url {http://www.ntu.edu.sg/home/EPNSugan/index_files/CEC2013/CEC2013.htm}}.\\

Uma vez que a integração entre Python e C é facilitada pelo uso da biblioteca {\it ctypes}, não foi necessário implementar nenhuma das oito funções. Somente foi necessário criar essa integração entre o código em C disponibilizado e a aplicação de GA em python. O método utilizado pode ser encontrado junto ao repositório Git do Peabox \footnote[6]{\url {https://github.com/stromatolith/peabox/tree/master/cec2013_testfuncs_via_ctypes}}.\\



\section{Análises GA - Time-slice Log-Likelihood Fitness Function}
Posteriormente, estendemos os estudos para o contexto da aplicação. Seguimos duas direções para termos objetivos comparativos: a primeira, testar a GA {\it Time-slice Log Likelihood Fitness Function} com as diferentes combinações de operadores; a segunda, testar o comportamento da mesma GA juntamente da técnica de pesos adaptativos dos operadores.\\

A técnica de pesos adaptativos utiliza valores variáveis para os operadores ao longo dos ensaios, valores que periodicamente são ajustados para refletirem a performance recente do operador. A maior justificativa para seu uso é: operadores podem mudar de importância ao longo do ensaio, e, portanto, sua influência deve ser alterada de acordo com o grau de importância adequado.\\ %Conseqüentemente, essa técnica possui elevadas chances de obter valores mais próximos dos ótimos.\\

Para a nossa aplicação, utilizamos operadores mais tradicionais, a fim de facilitar futuras análises e comparações. O operador de {\it crossover} utilizado foi o  {\it Two Points}, o de mutação, {\it Shuffle Indexes} e o de seleção, a roleta. A probabilidade inicial para o {\it crossover} foi de 0.9 e para a mutação, 0.1, valores arbitrários. Como inicialmente queremos abranger o espaço de busca o máximo possível, priorizamos o operador {\it crossover}. Ao final da execução, o espaço de busca já está bem definido, e, nesse caso, queremos especificar a busca, e, portanto, elevamos a prioridade de ocorrência de mutação para 30\% mais provável (conseqüentemente, precisamos diminuir a probabilidade do {\it crossover} em 30\%). Essa variação nos valores dos operadores acontece suavemente durante os ensaios, quando, a cada rodada, os valores dos operadores é alterado pelo valor = [30\%/(número de gerações)].\\


%\subsubsection{Dificuldades Enfrentadas}

%arrumar aqui por causa do profiler
Algumas adversidades foram encontradas durante as execuções, que contribuíram para o atraso da definição dos operadores. Entre elas, a de maior destaque foi a fraca performance do cálculo do fatorial, necessário para a função de {\it fitness}. Para melhor compreender o comportamento da aplicação e descobrir e confirmar quais eram gargalos da aplicação, foi utilizado um {\it profiler}\footnote[7]{\url {https://pythonhosted.org/line_profiler/}}.\\

Um {\it profiler} é uma ferramenta para análise dinâmica da execução de programas. A motivação é analisar o quanto de recurso computacional cada parte do código consome. Em específico, o {\it profiler} verifica a freqüência e a duração das chamadas do código. Após a descoberta dos pontos críticos e a confirmação de que o função de cálculo do fatorial realmente era um ponto crítico, substituímos a função por uma tabela em memória dos valores do fatorial.\\

%Ainda é possível que, a partir de análise de complexidade e reexecuções do {\it profiler}, outros gargalos sejam descobertos e tratados, melhorando a performance temporal da aplicação.\\

As análises realizadas com as funções CEC'13, ao contrário do esperado, não nos direcionaram na escolha de um grupo de operadores capaz de oferecer uma desempenho significativamente maior, seja ele em termos de convergência mais rápida ou em termos de soluções mais robustas.\\

%TODO: Add section about regions data, move it from the catalog data
%TODO: heat maps
%TODO: boxplot
%TODO: how to proceude forecast
%TODO:  how to evaluate
%TODO: 
  %\chapter{Resultados Obtidos}\label{chapter7}
\section{Simple L-test Fitness Function}
O operador escolhido como parâmetro de {\it crossover} foi o {\it Blend}, e a preferência por ele dentre os demais implementados pelo pacote DEAP foi pela observação empírica de crescimento dos valores de L-test e pelo fato de o operador ser específico para indivíduos formados por números reais, situação encontrada na aplicação. A média do resultado de 10 execuções:  \\
%arrumar dados
%O L-test do melhor indivíduo da população inicial é: 2.57371265858 e o valor mais alto de L-test obtido é: 6328.5844906

\begin{table}[!h]
  \begin{center}
  \begin{tabular}{|l|r|}
    \hline
    Primeira geração (s) & 145.6644\\
    Última geração (s) & 113.4796\\
    Valor do L-test da primeira geração & -0.635949214\\
    Valor do L-test da última geração & -0.009772415\\
    \hline    
  \end{tabular}
  \end{center}
  \caption{Tempo gasto e valor do L-test na média de 10 execuções com Blend.}
  \label{GAParameters--}
\end{table}

Operador de {\it crossover} {\it Two Points} foi utilizado por motivos de comparação. A média do resultado de 10 execuções: \\
%por em tabela
\begin{table}[!h]
  \begin{center}
  \begin{tabular}{|l|r|}
    \hline
    Primeira geração (s) & 129.03775\\
    Última geração (s) & 97.00832\\
    Valor do L-test da primeira geração & -0.635856248\\
    Valor do L-test da última geração & 0.042102222\\
    \hline    
  \end{tabular}
  \end{center}
  \caption{Tempo gasto e valor do L-test na média de 10 execuções com Two Points.}
  \label{GAParameters---}
\end{table}

Tanto o modelo preferido quanto o modelo comparativo foram capazes de evoluírem, obtendo valores finais médios superiores aos valores médios aleatórios iniciais.\\

Surpreendemente, por \cite{janikow1991experimental}, era esperado que representações em ponto flutuante tivessem um desempenho de maior acurácia. Porém, o desempenho do operador específico para números reais, o {\it Blend}, foi menor quando comparado a um operador de uso mais geral, o {\it Two Points}, sendo mais lento e obtendo valores de L-test menores, na média.	\\
	
Todos os dados foram obtidos após a execução do algoritmo em um computador Apple MacBook Pro com processador 2.9 GHz Intel Core i7, memória RAM 8 GB 1600 MHz DDR3 com sistema operacional OS X 10.9.1 (13B42).\\

%Pelo expressivo tempo total da aplicação, durante várias análises, utilizou-se um nú- mero menor de gerações (principalmente quando o objetivo era comparar outros trechos de código que não o trecho de evolução da população).\\


Para melhor visualização e aumentar o poder de comparação, quatro figuras podem ser analisadas a seguir. As Figuras \ref{popXmediaLtest_CXBLEND(1)} e \ref{popXmediaLtest_2POINTS(1)} mostram as médias do valores do L-test para a todas as populações enquanto que as figuras \ref{popXmediaG_CXBLEND(1)} e \ref{popXmediaG_2POINTS(1)} mostram as médias valores do tempo também para todas as populações. Figuras \ref{popXmediaLtest_CXBLEND(1)} e \ref{popXmediaG_CXBLEND(1)} são referentes a execuções com o {\it crossover Blend} e as figuras\ref{popXmediaLtest_2POINTS(1)} e \ref{popXmediaG_2POINTS(1)}, com o {\it crossover Two Points}. \\

%A mutação, FlipBit foi escolhida por estar presente nos exemplos fornecidos pelo pacote e ter sido facilmente entendida e pois dentre os operadores mutação implementados não foram de agrado, e futuramente deverá ser criada uma função específica para a aplicação.\\

Por observação empírica das diversas execuções realizadas o maior tempo gasto é com cálculos de L-test, sendo influenciado principalmente pela quantidade de observações e pelo tamanho escolhido para {\it bins}. Quanto menor for o tamanho escolhido para o {\it bin} maior será a resolução do terreno analisado e, conseqüentemente, mais informações sobre ele teremos e maior será o espaço de busca, aumentando o tempo total gasto.\\

Há um grande aumento do valor do L-test entre a vigésima geração e a quadragésima geração. Isso significa que a função de {\it fitness} resulta em um excessivo valor de {\it overfitting}, um super ajuste a base de dados.\\
%onde fica isso?
% Possivelmente, alterar a função utilizada pela mutação e alterar seu valor de acordo com a população deve ser capaz de minimizar a questão.



%Em um segundo momento, baseado nas observações anteriores, algumas modificações foram realizadas.
%%FIGURAS%%
%1
%\begin{figure}[!htb]
%\centering
%\includegraphics[scale=0.4]{popXmediaLtest_CXBLEND(1)}
%\caption{Média dos valores de L-test de cada população( com {\it crossover Blend.})}
%\label{popXmediaLtest_CXBLEND(1)}
%\end{figure}
%
%%2
%\begin{figure}[!htb]
%\centering
%\includegraphics[scale=0.4]{popXmediaLtest_2POINTS(1)}
%\caption{Média dos valores de L-test de cada população (com {\it crossover Two Points.})}
%\label{popXmediaLtest_2POINTS(1)}
%\end{figure}
%
%%3
%\begin{figure}[!htb]
%\centering
%\includegraphics[scale=0.4]{popXmediaG_CXBLEND(1)}
%\caption{Média tempo gasto (em segundos) por cada população, com {\it crossover Blend.}}
%\label{popXmediaG_CXBLEND(1)}
%\end{figure}
%
%%4
%\begin{figure}[!htb]
%\centering
%\includegraphics[scale=0.4]{popXmediaG_2POINTS(1)}
%\caption{Média do tempo gasto (em segundos) por cada população, com {\it crossover Two Points.}}
%\label{popXmediaG_2POINTS(1)}
%\end{figure}

\section{Time-slice Log Likelihood Fitness Function}

Os resultados da aplicação GA {it Log Likelihood  Fitness Function} serão demonstrados a seguir.\\

Foram feitas comparações entre os resultados dos valores de {\it fitness} dos melhores indivíduos e o desvio padrão da população a que ele pertence. Esses resultados comparados são referentes ao estudo dos grupos de operadores e a técnica de pesos adaptativos. A seguir, além dos resultados obtidos, algumas Figuras utilizadas para comparações serão mostradas.\\

As Figuras \ref{2000-media_melhores-1point} e \ref{2010-media_melhores-1point} demonstram o resultado das 50 execuções, mostrando dados referentes a execuções do conjunto de operadores {\it One Point, Worst} e {\it Shuffle Indexes} para os anos de 2000 (quando há um ganho de desempenho ao utilizar-se a tabela) e de 2010 (quando há uma perda de desempenho). As Figuras mostram o melhor indivíduo e o desvio padrão de sua população. Está claro que existe os resultados não são consistentes e que existe alguma falha na aplicação.\\

Essa falha ficou clara após a mudança do cálculo de fatorial pela tabela em memória. Até onde x = 27 (x se refere ao número da execução), os valores obtidos vieram de cálculos do fatorial, a partir disso, foi utilizada a tabela. Porém, devido a essa falha, não é possível qualificar se algum grupo gerou indivíduos mais aptos. Portanto é essencial corrigi-la para novas execuções sejam geradas e as comparações possam ser refeitas.\\

Foram duas as causas levantadas para essa falha. A primeira, refere-se ao valor limitante para a tabela do fatorial. Uma vez que a tabela possui valores até o fatorial de 100, quaisquer valores acima desse limitante terão seus fatoriais arredondados a 100. Há casos que essa aproximação foi vantajosa e a aplicação respondeu com modelos de melhor desempenho, mas em alguns casos, a aproximação foi desvantajosa.\\

A segunda, refere-se a um erro no código da execução. Após uma análise superficial do código, 
é provável que o mapeamento dos dados coletados esteja propagando algum erro. Esforços futuros serão direcionados para a correção deste erro.\\


%As Figuras \ref{2000-media_melhores-1point} e \ref{2010-media_melhores-1point} exemplificam o ocorrido, mostrando dados referentes a execuções do conjunto de operadores {\it One Point, Worst} e {\it Shuffle Indexes} para os anos de 2000 (quando há um ganho de desempenho ao utilizar-se a tabela) e de 2010 (quando há uma perda de desempenho). Até onde x = 27, os valores obtidos vieram de cálculos do fatorial, a partir disso, foi utilizada a tabela.\\

Ainda assim, a tabela em memória do valores do fatorial resultou em um avanço em termos de performance temporal. Uma vez que a segunda possível causa seja o verdadeiro erro da aplicação, acredita-se que o uso da tabela será capaz de facilitar futuras execuções.\\

Os resultados obtidos pela técnica de pesos adaptativos não foram tão elevadas quanto o esperado. Para efeitos comparativos dois grupos de figuras são mostrados a seguir. As Figuras \ref{2000-media_melhores} e \ref{2000-AUTO-media_melhores} mostram os valores dos melhores indivíduos e o desvio padrão da população para o ano de 2000 e compõem o primeiro grupo, já o segundo grupo é composto pelas figuras \ref{2010-media_melhores} e \ref{2010-AUTO-media_melhores} e segue o mesmo princípio, porém para o ano de 2010. No primeiro grupo, a Figura \ref{2000-media_melhores} refere-se aos dados da técnica em questão, enquanto que a outra Figura refere-se ao mesmo conjunto de operadores utilizados ({\it crossover}, {\it Two Points} e {\it Shuffle Indexes}), mas com pesos fixos. O segundo grupo é descrito da mesma maneira.\\

Não fica claro, pelos dados mostrados, quais são os benefícios que a técnica introduz para a aplicação em questão. Após os mais diversos tipos de análises, análises sobre a influência dos grupos de operadores, seja na própria aplicação GA ou mesmo nas funções da CEC'13 e análises sobre diferentes técnicas para estruturação da abordagem GA, não ficou claro qualquer tipo de variação substancial capaz de direcionar os estudos e definir uma abordagem única para os experimentos. Portanto, deve-se fazer a devida ponderação: será que as variações realmente influem pouco nos resultados da aplicação, ou a aplicação deve passar por refinamentos, que levem a resultados mais consistentes?\\

Para responder essa ponderação devemos analisar as escolhas feitas para a criação de nosso modelo. Pela própria estrutura definida pelo modelo proposto, cada {\it bin} é tratado independentemente dos seus vizinhos, não considerando a influência de sismos próximos. Porém, pela características inerentes aos sismos é claro que poucos são os casos de sismos completamente independentes entre si. Podemos considerar, logo, que a influência da independência influi consideravelmente e é ela a principal causa da mínima variação pré-citada. \\

Por outro lado, as funções CEC'13 são funções unicamente matemáticas e por isso a questão levantada anteriormente não é suficiente para sugerir que a resposta a nossa ponderação deva ser direcionada somente para o segundo ponto da pergunta, ``a aplicação deve passar por refinamentos''. Devemos deixar claro, entretanto, que essa independência entre os {\it bins} deve ser melhor explorada a fim de encontrar uma solução que possibilite considerar a dependência dos sismos. Nesse ponto, percebemos que a complexidade da questão é elevada. \\ 

Apesar da utilização da técnica de pesos adaptativos não ter contribuído para uma performance mais significativa, fomos capazes de observar que os valores médios do melhores indivíduos durante os anos foram muito mais coesos do que quando comparados aos valores similares dos ensaios com pesos fixos. Isso nos leva a crer que utilizar a técnica indiretamente influiu positivamente em nossos resultados. Consequentemente, a idéia de abranger uma área maior de busca e posteriormente especifica-la, mostrou-se, como esperado, bastante adequada.\\

Ainda em relação a técnica anterior, é possível que outras contribuições não tenham sido percebidas e que, de acordo com as devidas mudanças, sejamos capazes de percebê-las. Essas mudanças tanto podem ser algumas das já citadas como também na porcentagem de variação dos pesos dos operadores, que é de 30\% (escolhido arbitrariamente), nos valores desses pesos na situação inicial, entre outras, ou seja, variações na estrutura utilizada juntamente com a técnica.\\


%%%FIGURAS%%%%
%5
%\begin{figure}[!h]
%\centering
%\textbf{Ano 2000}\par\medskip
%\includegraphics[scale=0.5]{2000-media_melhores-1point.pdf}
%%\small
%\caption{Melhor indivíduo e desvio padrão da população(y) em cada execução(x), desempenho médio superior com a tabela do fatorial.}
%\label{2000-media_melhores-1point}
%\end{figure}
%
%%6
%\begin{figure}[!h]
%\centering
%\textbf{Ano 2010}\par\medskip
%\includegraphics[scale=0.5]{2010-media_melhores-1point.pdf}
%\caption{Melhor indivíduo e desvio padrão da população(y) em cada execução(x), desempenho médio inferior com a tabela do fatorial.}
%\label{2010-media_melhores-1point}
%\end{figure}
%
%%7
%\begin{figure}[!h]
%\centering
%\textbf{Ano 2000}\par\medskip
%\includegraphics[scale=0.5]{2000-media_melhores.pdf}
%\caption{Melhor indivíduo e desvio padrão da população(y) em cada execução(x), pesos fixos.}
%\label{2000-media_melhores}
%\end{figure}
%
%%8
%\begin{figure}[!h]
%\centering
%\textbf{Ano 2000}\par\medskip
%\includegraphics[scale=0.5]{2000-AUTO-media_melhores.pdf}
%\caption{Melhor indivíduo e desvio padrão da população(y) em cada execução(x), GA com pesos adaptativos.}
%\label{2000-AUTO-media_melhores}
%\end{figure}
%
%%9
%\begin{figure}[!h]
%\centering
%\textbf{Ano 2010}\par\medskip
%\includegraphics[scale=0.5]{2010-media_melhores.pdf}
%\caption{Melhor indivíduo e desvio padrão da população(y) em cada execução(x), pesos fixos.}
%\label{2010-media_melhores}
%\end{figure}
%
%%10
%\begin{figure}[!h]
%\centering
%\textbf{Ano 2010}\par\medskip
%\includegraphics[scale=0.5]{2010-AUTO-media_melhores.pdf}
%\caption{Melhor indivíduo e desvio padrão da população(y) em cada execução(x), GA com pesos adaptativos.}
%\label{2010-AUTO-media_melhores}
%\end{figure}

There are 3 tests hypotheses for this experiment that we would like to check. 

The first is if the mean values of the log-likelihood for the ReducedGA are equal to the RI values.
$$\begin{cases} H_0: \mu = RI_log-likelihood_value&\\H_1: \mu != RI_log-likelihood_value\end{cases}$$

The second is if the mean values of the log-likelihood for the ReducedGA are equal to the GAModel values.
$$\begin{cases} H_0: \mu = GAModel_log-likelihood_value&\\H_1: \mu != GAModel_log-likelihood_value\end{cases}$$

And the last hypotheses is if the mean values of the log-likelihood for the GAModel are equal to the RI values.
$$\begin{cases} H_0: \mu = RI_log-likelihood_value&\\H_1: \mu != RI_log-likelihood_value\end{cases}$$
Summary
 
O objeto é descobrir se existem variações ente os métodos e quais são as variáveis mais influentes.

%Os métodos utilizados para comparação são o gaModel,a versão com listas, os sistemas híbridos (hybrid_gaModel e hybrid_lista). Para cada um dos métodos temos algumas variações nas varíaveis utilizadas. Variamos os anos (2005-2010), as regiões (Kanto, EastJapan, Touhoku e Kansai), a profundidade ( <25km, <60km, <100km) e finalmente o catálogo utilizado (JMA X métodoJanelaJMA=>clustered).

STAtistical Analysis
ANOVA test and HSD Tukey

Vou utilizar o ANOVA para nos dados obtidos para verificar qual composição de variáveis e métodos mais influênciam no resultado final.

%Para isso executei o *gaModel*, *versão com Listas*, *hybrid_gaModel* e *hybrid_lista* para cada conjunto de variáveis 10 vezes. Cada grupo para um método é composto por: região, ano, profundidade e catálogo. Um grupo para um cenário será chamado cenário de execução.
%
%Após as execuções vou aplicar o ANOVA em uma data.frame composto pelos dados das **médias dos melhores indivíduos da última geração** para cada cenário de execução. 
%
%Caso uma variável esteja fora do intervalo de confiança (P < 0.05), vou aplicar novamente o ANOVA retirando essa variável do teste. 

Aplico um teste post hoc nos resultados do ANOVO oara especificar quais são os grupos que diferem. O teste utilizado foi o Tukey teste.

É importante resaltar que para todos os casos, aplico uma função de limite, que altera os valores do bins com mais que 12 ocorrências para 12.

Começo a análise carregando o data.frame com os dados, seguindo para a aplicação do teste ANOVA e finalizando com o uso do Tukey teste.


Faço o ANOVA somente para os modelos "clusterizados"

Aplico o anova, com a regressão para modelos, profundidades, anos e regiões.

Agora faço o Paired Design t.test aplicando para todas as combinações possíveis de modelos, em todas as regiões e profundidades, para todos os anos.

Baseado nos arquivos que explicam o Paired Desing, escrevi o código a seguir. Porém não entendi porque ao fazer desta forma pode ser considerado um teste pareado. Os slides comparam duas formas de realizar este tipo de teste. Uma delas tem 
*seta* um parametro da função com **True**, explicitando que é um teste pareado. Já para o outra forma, esse parametro fica com **False**.

%A one-way between subjects ANOVA was conducted to compare the effects of the models, the depths, the years and regions on the log-likelihood value. In this study there are 6 options for model: lista, gaModel, hybrid_gaModel, hybrid_list, gaModelCluster and listaCluster. Based on the results of the test, there was a not a significant effect of the depths or years variables. For both cases at the we obtaind p>0.05 level for the depths condition [F(2) = 2.072, p = 0.126] and we also obtained p>0.05 for the years condition  [F(5) = 0.050, p = 0.999]. There was a significant effect of the models condition (p>0.05 [F(5) = 9699.690, p<2e-16]) and regions condition (p>0.05 [F(3) = 764.220, p<2e-16]). Therefore, we conduct a new anova test, with only the last two variables to verify the influence of those conditions more accurately. The results only changed a little, maintaining the significant effect of both conditions, p>0.05 [F(5) = 9705.6, p<2e-16] and p>0.05 [F(3) = 764.7, p <2e-16], respectively.
%
%Because we found statistically significant result, we applied a Post hoc comparisons using the Tukey HSD test. It compared each condition with all others. For example, it compares the values from the gaModel with the gaModelClustered. It indicated that the gaModelCluster and the listaCluster, when comparared with all other models, achieve greater log-likelihood values. Furthermore, we noticed that the depths conditions show a greater influence when the depth in smaller or equal to 25 km.

When comparing the models from the lista method and from the gaModel against themselves, with or without using clustering techniques, we found that there is no statistically significant result between the methods. That implies that it can be considered that the methods are obtain statistically equal results.

Therefore, based on the result of the HSD test, we performed a new AVOVA test, considering only the gaModelClustered and the listaClustered. That was meant not only to verify the previous results but also to certify if the depth influence is preserved.

Taken together, these results suggest that the using cluster and depth smaller or equal to 25km showed the best results. 
%TODO: add paired test results
  % ...

  \nocite{*}
  \postextual
  \bibliographystyle{plain}
  \bibliography{bibliografia}

\end{document}
