%\chapter{State of Art}~\label{chapter4}
In this chapter we will briefly about Genetic Algorithms and then discuss some reports of the application of Evolutionary Computation and related method for Earthquake Risk Analysis.\\

\section{What are Genetic Algorithms}
The main goal of a Genetic Algorithm (GA) is to find approximated solutions in problems of search and optimisation Based on Koza~\cite{koza2003genetic}, GA are mechanism of search based on natural selection and genetic. They explore historical data to find optimum search points with some performance increment, as said by Goldberg~\cite{Goldberg:1989:GASac:534133}.\\

\subsection{How does GA work}

A GA uses those mechanisms to generate solutions to optimisation and search problems. The first step is to create an initial population of possible solutions. Frequently, the initial population is randomly generated once it is common to ignore the main aspects that influence the algorithm performance.\\

Each possible solution of a population is called an individual. Every individual is a possible solution of a problem. Those individuals have its fitness value estimated by a fitness function. A fitness function should determine how suitable a individual is to a given problem. The most suitable individuals are graded with better values and the not so suitable ones have a lower value.\\

After measuring the population fitness value, some individuals are then selected by a process that takes into account each individual fitness value to influence the next population. The individuals with better values have a higher chance to be selected. The individuals selected take part in the variation process. This process may alter some of the individual characteristics using the crossover and mutation operators.\\

The crossover operator is a operator that is used to vary the characteristics of a group of individuals. For that a number of parents, a group of individuals from the current population, are selected. In most of the cases, the parents are chosen to compose a pair that will exchange information that will take compose the child, a new individual that will belong to the next generation.\\

Another important operator is called the mutation operator. It is a operator with the purpose of avoiding the loss of important information. It works by changing the characteristics of an individual, looking to add new information to the next population.\\

It is common to have a evolutionary operator that allows the fittest individual from the current generation to take part in the next generation. This operator is called Elitism and it is used to assure that the next generation best solution is at least as good as in the current generation.\\

\section{Evolutionary Computation and Earthquake Risk Prevision}
%ANN
The usage of Evolutionary Computation in the field of earthquake risk models is somewhat sporadic.\\

Zhang and Wang~\cite{Zhang2012}, Zhou and Zu~\cite{Feiyan2014} and Sadat~\cite{sadat2015application}, used Artificial Neural Network (ANN) related with earthquake prediction. In all these works, they combine the ANN with some other technique, to achieve better results. They used a group of earthquake parameters, as the accumulated release energy, magnitude in a specific area, the b-value and others. Some parameters in this group are not available in our earthquake database.\\

Those papers use the available parameters of the earthquakes that happened in the area of study to create a risk prediction of earthquake or to propose a magnitude range for future earthquakes. They object to consider each variables influence the most the results so that their methods can achieve higher performance. We may compare and/or evaluate our method by comparing it to the works cited before.\\

%Fault Model parameters
%TODO: rewrite
Nicknam et al.~\cite{Nicknam2010} simulated some components from a seismogram station and predicted seismograms for other stations. They combined the empirical Green’s function (EGF) with GA. the EGF method is used to synthesise acceleration time histories and the GA approach is developed to optimise the seismological model. They found that this method obtained good agreement with the observed data, but are not sure that results are free from uncertainties.\\

In this paper, they work with more than 30 seismological model parameters. Although, that amount of parameters is not available to us, we can use the information from this paper to exam two options. The first, we may investigate if more earthquake parameters will improve our method and the other option is to analyse how they dealt with so many variables. Then we may consider to do the same and observe the results.\\  

Kennett and Sambridge~\cite{Kennett1992} used GA and associated teleseisms procedures to determine the Fault Model parameters of an earthquake. By doing so, they demonstrated that non-linear inversion can be achieved for teleseismic problems without any calculation of waves travel times. They used only P-wave data and expect that if more data could be introduced, the method would accomplish better results.\\

%PGA
Some seismological models were developed aiming to estimate parameter values by using Evolutionary Computation. For example, Evolutionary Computation was used to estimate the peak ground acceleration (PGA) of seismically active areas~\cite{Kermani2009,Cabalar2009,Kerh2010,Kerh2015}. \\

The works done by Kerh~\cite{Kerh2010, Kerh2015} are basely a combination of ANN and GA to estimate or predict PGA in Taiwan. These work are based on the benefits of mixing both techniques. They state that the usage of a purely ANN method to estimate PGA may fall into a local minimum and that can be avoid by combining ANN with GA, hence GA is a good method to find global optimums.\\

Their goal was to decide which areas may be considered potentially hazardous areas. They focused on urban areas, these works are important to revalidate building regulations, urban development and such. The earthquake variables that were used in these work are: local magnitude, focal depth, and epicentre distance. Both magnitude and depth are already used in our work, which is not the case of the epicentre distance variable. They also state that PGA is inversely proportional to epicentre distance, so to add data about this variable may be useful to our work, once it could provide useful information to predict risk models both direct or indirectly.\\
 
Ramos~\cite{Ramos2011} used Genetic Algorithms to decide the location of sensing stations. In this work Ramos achieved, in general, better results with the GA method when compared with the seismic alert system (SAS) method and a greedy algorithm method. In some cases, the SAS has a better response time than the GA. They consider it to be once caused because the SAS only alerts when earthquakes with magnitude bigger than 5.0 degrees in the Richter scale occurs, while the GA deals with all the earthquakes.\\

Ramos's work is a important work because it helps the population to avoid bigger disasters caused by earthquakes by increasing the time response of the Seismic-Sense Stations. It has some similar feature as the one present in this document: it uses GA to prevent earthquake disasters and tries to locate targets in a given area (though the targets of this work is sensing stations and ours works target is the earthquakes location) and it proposes a methodology to do a GA parameter setting to find which combination of values for the GA parameters achieve higher results. It is interesting to state that once a solution places a station in an area that is not possible to be have sensors, this possible solution suffers some penalties\\

Saeidian~\cite{saeidian2016evaluation} also based on the same idea of locating sensing stations. His work differs from the work of Ramos because it makes a comparison in performance between the GA and Bees Algorithm (BA) to decide which of those techniques would perform better when choosing the location of sensing stations. He found out that the GA was faster than the BA.\\

Huda and Santosa \cite{ijse5762} published a paper in which the goal is to find, via Genetic Algorithm, the speed of the waves P and S in the mantle and in the earth crust. P waves are indicated as the first fault found  in seismological data and S waves are the changes caused in the phase of a P wave~\cite{ijse5762}. This research aims to obtain a structure of the Japanese underground and geographically focuses in the same region as our work, though it uses data from two kinds of waves which are not available to us.\\