%\chapter{State of Art}~\label{chapter4}
In this chapter we will briefly about Genetic Algorithms and then discuss some reports of the application of Evolutionary Computation and related method for Earthquake Risk Analysis.\\

\section{What are Genetic Algorithms}
The main goal of a Genetic Algorithm (GA) is to find approximated solutions in problems of search and optimization. Based on Koza~\cite{koza2003genetic}, GA are mechanism of search based on natural selection and genetic. They explore historical data to find optimum search points with some performance increment, as said by Goldberg~\cite{Goldberg:1989:GAS:534133}.\\

\subsection{How does GA work}

A GA uses those mechanisms to generate solutions to optimization and search problems. The first step is to create an initial population of possible solutions. Frequently, the initial population is randomly generated once it is common to ignore the main aspects that influence the algorithm performance.\\

Each possible solution of a population is called an individual. Every individual is a possible solution of a problem. Those individuals have its fitness value estimated by a fitness function. A fitness function should determine how suitable a individual is to a given problem. The most suitable individuals are graded with better values and the not so suitable ones have a lower value.\\

After measuring the population fitness value, some individuals are then selected by a process that takes into account each individual fitness value to influence the next population. The individuals with better values have a higher chance to be selected. The individuals selected take part in the varation process. This process may alter some of the individual characteristics using the crossover and mutation operators.\\

The crossover operator is a operator that is used to vary the characteristics of a group of individuals. For that a number of parents, a group of individuals from the current population, are selected. In most of the cases, the parents are chosen to compose a pair that will exchange information that will take compose the child, a new individual that will belong to the next generation.\\

Another important operator is called the mutation operator. It is a operator with the purpose of avoiding the loss of important information. It works by changing the characteristics of an individual, looking to add new information to the next population.\\

It is common to have a evolutionary operator that allows the fittest individual from the current generation to take part in the next generation. This operator is called Elitism and it is used to assure that the next generation best solution is at least as good as in the current generation.\\

\section{Evolutionary Computation and Earthquake Risk Prevision}
%ANN
The usage of Evolutionary Computation in the field of earthquake risk models is somewhat sporadic.\\

Zhang and Wang~\cite{Zhang2012} used Genetic Algorithms to fine tune an Artificial Neural Network (ANN) and use this system to produce a forecast model. They integrate the global searching from the GA with the local searching ability of the ANN proposed a new model called GA-BP ANN. It optimize the initial weights and thresholds of the ANN and then, it trains the ANN. They compared the new model with its older version and found that the GA-BP ANN can make better network configurations and can improve the efficiency, precision and stability of earthquake risk prediction.\\

Zhou and Zu~\cite{Feiyan2014} make a very similiar work as the one proposed by Zhang and Wang. In this case they proposed a combination of BP ANN with the Levenberg–Marquardt algorithm and their system only forecasts the magnitude parameter of earthquakes. Sadat, in the paper~\cite{sadat2015application}, follows the idea of Zhou and Zu, aiming to predict the magnitude of the earthquakes in North Iran, but in this case, he used ANN and GA.\\

All these three work are based on using the available characteristics of the earthquakes that happen in the area of study to create a risk prediction of earthquake or to propose a magnitude range for future earthquakes. They object to consider each variables influence the most the results so that their methods can achieve higher performance. These works may help us to compare and evaluate our method, or part of it.\\

%Fault Model parameters
Nicknam et al.~\cite{Nicknam2010} simulated some components of a seismogram a station and predicted seismograms for another station. They combined  empirical Green’s function (EGF) with GA. the EGF method is used to synthesize acceleration time histories and the GA approach is developed to optimize the seismological model. They found that this method obtained good agreement with the observed data, but are not sure that results are free from uncertainties.\\

In this paper, they work with more than 30 seismological model parameters. We can use this information for two paths of action. The first, we may investigate if more earthaquake parameters will improve our method and the other path is to analyse how they dealt with some many variables. Then we may consider to do the same and inspect the results.\\  

Following the same idea proposed by the paper comment above in the work done by Kennett and Sambridge~\cite{Kennett1992}. They also used GA and associated teleseisms procedures to determine the Fault Model parameters of an earthquake. By doing so, they demonstrated that non-linear inversion can be achieved for teleseismic problems without any calculation of waves travel times. They used only P-wave data and expect that if more data could be introduced, the method would accomplish better results.\\

%PGA
Some sismological models were developed aiming to estimate parameter values by using Evolutionarry Computation. For example, Evolutionary Computation was used to estimate the peak ground acceleration (PGA) of seismically active areas~\cite{Kermani2009,Cabalar2009,Kerh2010,Kerh2015}. \\

The two works done by Kerh~\cite{Kerh2010, Kerh2015} are basely a combination of ANN and GA to estimate or predict PGA in Taiwan. These work are based on the benefits of mixing both techniques. They state that the usage of a purely ANN method to estimate PGA may fall into a local minimum and that can be avoid by combining ANN with GA, hence GA is a good method to find global optimums. By doing that the new ANN+GA method will achieve more reliable results.\\

They aimed to decide which areas may be considered potentially hazardous areas. Hence they focus on urban areas, these works are important to revalidate building regulations, urban development and such. The earthquake variables that were used in these work are: local magnitude, focal depth, and epicenter distance. Both magnitude and depth are already used in our work, which is not the case of the epicenter distance variable. They also state that PGA is inversely proportional to epicenter distance, so this variable may be useful to our work by adding useful information to predict risk models both direct or indirectly.\\
 
Ramos~\cite{Ramos2011} used Genetic Algorithms to decide the location of sensing stations. In this work Ramos achieved, in general, better results with the GA method when compared with the seismic alert system (SAS) method and a greedy algorithm method. In some cases, the SAS has a better reponse time than the GA. They consider it to be once caused because the SAS only alerts when earthquakes with magnitude bigger than 5.0 degress in the Richter scale occurs, while the GA deals with all the earthquakes.\\

Ramos's work is a important work because it helps the population to avoid bigger disasters caused by earthquakes by incrensing the time response of the Seismic-Sense Stations. It has some similar feature as the one present in this document: it uses GA to prevent earthquake disasters and tries to locate targets in a given area (though the targets of this work is sensing stations and ours works target is the earthquakes location) and it proposes a way to do GA parameter setting to find which combination of values for the GA parameters achieve highier results. It is interesting to state that once a solution places a station in an area that is not possible to be have sensors, this solution suffers some penalities.\\

Saeidian~\cite{saeidian2016evaluation} also based on the same idea of locating sensing stations. His work differs a little when compared with one above because it makes a comparation in performance between the GA and Bees Algorithm to decide which of those techniques would performe better when chosing the location of sensing stations.\\

Huda and Santosa \cite{ijse5762} published a paper in which the goal is to find, via Genetic Algorithm, the speed of the wavesP and S in the mantle and in the earth crust. P waves are indicated as the first fault found  in seismological data and S waves are the changes caused in the phase of a P wave~\cite{ijse5762}. This research aims to obtain a structure of the japanese undergound and geographically focuses in the same region as our work.\\