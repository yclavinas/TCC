Entender os mecanismos e padrões dos terremotos é importante para minimizar suas consequências. Neste contexto, este projeto visa desenvolver um modelo de previsão de riscos de terremotos com Algoritmos Geneticos (GA). Modelos de risco de terremotos descrevem o risco de ocorrência de atividades sísmicas in uma determinada área baseado em informações previamente obtidas de terremotos em regiões próximas da área de estudo. Nós utilizamos GA para aprender um modelo de risco usando somente informações previamente obtidas como base de treino. Baseado nos resultados obtidos, nós acreditamos que é possível obter melhores models se conhecimento do domínio da aplicação, como conhecimentos oriundos da literatura ou modelos de distribuição de terremotos, poderem ser incorporados ao processo de aprendizado do Algoritmo Evolutivo.

  O objetivo principal é definir um método para estimar a probabilidade de ocorrências de terremotos no Japão usando dados históricos de terremotos para um grupo de determinadas regiões geográficas. Este trabalho se baseia no contexto do  “Collaboratory for the Study of Earthquake Predictability” (CSEP), que visa padronizar os estudos e testes de modelos de previsão de terremotos. 
  
  Durante o desenvolvimento das atividades, passamos por quatro estágios. (1) Nós propusemos um método baseado em uma applicação de Algoritmos Genéticos (GA) e objetivamos gerar um método estatístico de análise de risco de terremotos. Estes foram analisados por seus valores de \textit{log-likelihood}, como sugerido pelo \textit{Regional Earthquake Likelihood Model} (RELM). (2) Basedos nos resultados obtidos, nós buscamos melhorar a performance do GA ao incluir uma técnica de GA chamada auto-adaptativo. (3) A seguir, modificiamos a representação do genoma, de uma representação baseada em área para uma representação baseada em ocorrências de terremotos, buscando obter uma convergência mais rápida dos valores de \textit{log-likelihood} dos candidatos do GA e (4) usamos métodos da sismologia conhecidos (como a equação de Omori-Utsu) para refinar os candidatos gerados pelo GA.
  
Em todas as estapas, os modelos de risco são comparados com dados reais, com modelos gerados pela aplicação do \textit{Relative Intensity Algorithm} (RI) e com eles próprios. Os dados utilizados foram obtidos pela \textit{Japan Metereological Angency} (JMA) e são relativos a atividades de terremotos no Japão entre os anos de 2000 e 2013.

Nós analisamos as contrubuições de cada modelo proposto usando metodologia descritas pelo CSEP e comparamos as performances entre (XXX method and YYY method). Os resultados apontam (XXX result, YYY result).
/*terminando com os resultados obtidos e conclusoes alcancadas.*/