%\chapter{Metodologia Proposta}\label{chapter6}
\section{The GAModel}
We used the  Log-likelihood value~\ref{log-fuction} as the fitness function. For the experiments, We divided the data into annual slices, as described in the chapter~\ref{capitulo5}, for the region of Kanto.\\

Once no effort was made to analyse the GA initial population and number of generations, we chose them by trial and error, until an acceptable convergence time were achieved.\\

\section{The GAModel Experiment}
	We create some scenarios to compare the GAModel with the RI - Relative Intensity Algorithm.\\
	
	These scenarios are defined as space/time regions. Each scenario contains the earthquakes for the Kanto region for a given year.\\
	
	As being a stochastic method we used the Power Test and estimated the number of repetitions, $n$, needed for detecting a significant variation. For the  {\it Power Test} we used a pilot experiment to estimate $n$.\\
	
	An example of the  {\it Power t-test.}:

\begin{table}[!h]
  \begin{center}
  \begin{tabular}{|l|r|}
%  	\head One-sample t test power calculation 
    \hline
    Year & 2006\\
    Delta & 50 (for all scenario)\\
    Standard Deviation &  25.16513\\
    Degree of confidence & 0.05\\
    power & 0.95\\
    alternative & two.sided\\
    \hline
    $n$: & 5.58517\\
    \hline    
  \end{tabular}
  \end{center}
  \caption{Power t-test.}
  \label{power}
\end{table}

The pilot experiments used 10 observations for calculating the {\it Power Test}. The $delta$ and $power$ values were chosen based on the results of the pilot experiment. The standard deviation is the same as the observations. All results indicate that 10 repetitions are enough to compare the GAModel via {\it Student's t-test}.\\

We also needed the log-likelihood of the RI method. As being a deterministic method, 1 observation for each scenario is enough. This value is used as the target for the {\it Student's t-test}. For control, we used a random Poison method with no data awareness.\\

\subsection{The Relative Intensity Algorithm}\label{ri}
The RI {\it Relative Intensity} (RI) algorithm is frequently used as reference for comparing methods~\cite{Nanjo2011}. The main idea behind the RI is that larger earthquakes are more likely to occur at locations of high seismology in the past.\\
	
The log-likelihood data for the RI for each scenario were given by Aranha,~\cite{ecta14}.\\



\subsection{Evolutionary Operators}

The GAModel use a combination of operators made available by the Distributed Evolutionary Algorithms in Python (DEAP)~\cite{DeRainville}. We used the One Point Crossover for the crossover operator, the Polynomial Bounded Mutation for the mutation operator and for selection, we used Tournament selection and Elitism. The parameters are described in the Table~\ref{GAParameters}.\\

\begin{table}[!ht]
	\caption{Parameters used in GAModel}
	\label{GAParameters}
	\begin{center}
		\begin{tabular}{|l|r|}
			\hline
			Population Size & 500\\
			Generation Number & 100\\
			Elite Size & 1\\
			Tournament Size & 3\\
			Crossover Chance & 0.9\\
			Mutation Chance (individual) & 0.1\\
			Polynomial Bounded parameters & beta = 1, low = 0, up = 1\\
			\hline    
		\end{tabular}
	\end{center}
\end{table}



\subsection{The GAModel and The RI Algorithm}
We compared the method proposed, ReducedGAModel,~\ref{ReducedGAModel}, with the non-variant method, GAModel~\ref{GAModel}, and with the Relative Intensity Algorithm,~\ref{ri}, using its log-likelihood values. The values are compared via Student’s t-test, so we could understand the if there is any statistic significant difference between the methods. The data used was JMA catalogue\\

\subsection{Hypothesis}
There are 3 tests hypothesis for this experiment that we want to analyse. For all, the confidence interval is set to 5\%.\\

The first is if the mean values of the log-likelihood for the ReducedGAModel are equal to the RI values.
$$\begin{cases}
	H_0: \mu = \text{RI log-likelihood value}&\\
	H_1: \mu != \text{RI log-likelihood value}
\end{cases}$$

The second is if the mean values of the log-likelihood for the ReducedGAModel are equal to the GAModel values.
$$\begin{cases}
	H_0: \mu = \text{GAModel log-likelihood value}&\\
	H_1: \mu != \text{GAModel log-likelihood value}
\end{cases}$$

And the last hypotheses is if the mean values of the log-likelihood for the GAModel are equal to the RI values.

$$\begin{cases}
	H_0: \mu = \text{RI log-likelihood value}&\\
	H_1: \mu != \text{RI log-likelihood value}
\end{cases}$$


\section{The Models}

Based on our promising results and because we aim to improve them, we developed the ReducedGAModel. It a simplified version of the GAModel. We want to compare the behaviour of this new method against the GAModel method.\\

We also wanted to explore how adding domain knowledge would improve the average performance of the GAModel or the ReducedGAModel They are versions of GAModel/ReducedGAModel combined with empirical laws.\\ 

\section{The Main-shock Models and Main-shock with After-shock Method Experiments}
 Here we describe the catalogues and the evolutionary operators used for the experiments and some examples of the models generated. Then we specify the models comparison.\\
 
\subsection{The catalogues}

The data used was JMA catalogue, with the magnitude minimum of 3.0 in the Richter Scale and the two de-clustered catalogues, obtained from the methods explained in the Chapter~\ref{chapter5}, Section~\ref{Clustering}.\\ 

\subsection{Evolutionary Operators}

For the ReducedGAModel the only different operator is the mutation function,~\ref{GAParameters}. We use a simple mutation operator which samples entirely two new values, both sampled from uniform distributions. The first, is a new real value from [0,1) and the second one, a new integer value from [0,X), where X is the maximum length of the genome. For the parameters see Table~\ref{GAHParameters}.\\

\begin{table}[!ht]
	\caption{Parameters used in ReducedGAModel}
	\label{GAHParameters}
	\begin{center}
		\begin{tabular}{|l|r|}
			\hline
			Population Size & 500\\
			Generation Number & 100\\
			Elite Size & 1\\
			Tournament Size & 3\\
			Crossover Chance & 0.9\\
			Mutation Chance (individual) & 0.1\\
			\hline    
		\end{tabular}
	\end{center}
\end{table}

The Emp-GAModel~\ref{emp-gamodel} uses the same operators as the GAModel. The parameters are described in the Table~\ref{GAParameters}.\\

For the Emp-ReducedGAModel~\ref{emp-reducedgamodel} uses the same operators as the ReducedGAModel. The parameters are described in the Table~\ref{GAHParameters}.\\


\subsection{Models Comparison}
For this new experiment, we used even more scenarios (space/time regions) than the others. Each scenario contains the earthquakes for the regions of Kanto, Kansai, Touhoku and East Japan for a given year (2005-2010). We wanted to explore if there exists any influence in the performance of the models that are caused by the depth of an earthquake. So, the scenarios are also composed by introducing a 3 groups depths thresholds. They are: of earthquakes with depth smaller than 25km, or between 0km and 60km or even between 0km and 100km.\\

These method are stochastic method and hence are variations of the GAModel, we decided to maintain the number of repetitions without redoing the Power Test.\\

\subsection{Statistical Analysis Of The Results}

The goal is to discover if there is any variation between the methods and which are the most influential variables. For achieve that, we will use the ANOVA Test.\\

In the ANOVA test, if a variable is out of the 5\% confidence interval, with P < 0.05 it means that there exists a statistical significant different for that variable.\\

There are some tests hypothesis for this experiment that we want to analyse. They all can be generalised as follows:\\

$$\begin{cases} H_0: \text{The population means are equal populacionais são iguais.} &\\
H_1: \text{The population means are different.}&\\
\end{cases}$$\\

Then we apply a post hoc test, the HSD Tukey test, on the results obtained from the ANOVA test to specify which groups differ. \\

Tukey's test compares the means of a case with the means of every other case. Doing so, it identifies differences between means:\\

$$\begin{cases}
\mu_a-\mu_b &\\
\end{cases}$$\\
