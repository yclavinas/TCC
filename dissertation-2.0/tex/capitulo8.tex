%TODO: add contributions and future works

Este trabalho procurou responder questões sobre a possibilidade de utilização de algoritmos
genéticos para realizar a tarefa de projetar circuitos digitais. Enquanto um
dos objetivos principais, o estudo de soluções sequenciais, não pôde ser realizado devido
aos problemas técnicos descritos, resultados interessantes surgiram mesmo apenas para o
projeto de circuitos combinacionais.

In this work we proposed some methods to generate earthquake risk model. We described how they were built and their characteristics. All models were based on the GAModel, and differ from it in the genome representation, the use of seismological equations or the catalogue used. \\

Initially, we studied the CSEP framework.  Based on it we proposed a GA method to generate a earthquake risk model. We implemented the method and did some experiments comparing it with the model from the RI Algorithm. The scenario used for this experiments was composed by the years of 2000 to 2010 for the region of Kanto. The results showed that this method is competitive with the RI Algorithm.\\

Supported on this results and because we wanted to improve the performance of the method in terms of log-likelihood values, we proposed two methodologies. The first, is to change the genome representation. The other is to use seismological equations that would improve the accuracy of the methods by adding domain awareness to them.\\

The main goal of changing the genome representation is to minimise the search space. We expected that by doing so, the GA would have to consider less possibilities and would find a good solution with less computational effort and/or could lead to models with higher log-likelihoods models.\\

The new representation achieved similar results as the one from the GAModel. Hence we expected to achieve greater log-likelihood values than the GAModel, we consider that more efforts should be direct to refine this representation. We think that this representation would benefit if we consider areas that contained more than one bin, instead of how it is now. Because it would have some flexibility in positioning the earthquakes.\\

Secondly, we studied the ETAS model and the seismological equations that it uses. That lead to the study of the Omori-Utsu formula and some others related formula. From this study, we encountered many difficulties, mostly related to lack of earthquake background knowledge. Although, during this study we realised that it would be interesting to analyse how the depths influence our methods and to consider a classification of the earthquakes into main-shocks and after-shocks. \\

The usage of the seismological equations show no improvement to the models. We think, though, that there is still space to improve the usage of these equations. If we better understand how the seismological equations behave and when is the best time to use them, probably this would lead to some improvement to the models. Also, we may want to try some other group of equations, once we focused on the Omori-Utsu formula and its relative equations.\\

We compared the models generated for each scenario. From the statistical analysis, we could analyse how the earthquakes variables influence the models and which combination result in models with higher log-likelihood values.\\

From the analysis, we discovered that more stable earthquake are easier to predict. These earthquakes have depth smaller or equal to 25km, magnitude higher than 4.0 in the Richter Scale. We also discovered that to classify earthquakes into main-shocks and after-shocks improve our methods predicting ability.\\

After considering all analysis, the method that achieve better results in the statistical analysis, was the \textit{GAModelWindow}. We propose a comparison of this method models with the RI Algorithm and the GAModel to proof test this method.\\


%hibridizacao pra diferenciar o modelo
%One way to mitigate the sharpness noticed in the results is by making the algorithm aware of data lo- cality. 
%quais terremotos são mais legais :D
%qual modelo é mais legal
%
