To understand the mechanisms and patterns of the earthquakes is very important to minimize its consequences. In this context, this projects aims to develop an earthquake prevision risk model using Genetic Algorithms (GA). Earthquake Risk Models describe the risk of occurence of seismic events on a given area based on information such as past earthquakes in nearby regions, and the seismic properties of the area under study. We used EC to learn risk models using purely past earthquake occurrence as training data. Based on the results obtained, we believe that a much better model could be learned if domain knowledge, such as known theories and models on earthquake distribution, were incorporated into the Genetic Algorithm's training process.

The main goal is to define good methods to estimate the probability of earthquake occurrences in Japan using historical earthquake data of a group of given geographical regions. This work is established in the context of the “Collaboratory for the Study of Earthquake Predictability” (CSEP), which seeks to standardize the studies and tests of earthquake prevision models. 

To achieve the main we passed four stages. (1) We proposed a method based in one application of Genetic Algorithms (GA) and aims to develop statistical methods of analysis of earthquake risk. The risk models generated by this application were analyzed by their log-likelihood values, as suggested by the Regional Earthquake Likelihood Model (RELM). (2) Based on the results obtained, we tried to improve the GA's performance by including the self-adaptive GA technique. (3) Then, we modify the genome representation from an area-based representation to an earthquake representation aiming to reach a faster convergency of the log-likelihood values of the GA's candidates and (4) we use known methods from seismology (such as the Omori-Utsu formula) to refine the candidates generated by the GA.

In all stages, the risk models are compared with real data, with the models generated by the application of the Relative Intensity Algorithm (RI) and with themselves. The data used was obtained from the Japan Metereological Angency (JMA) and are related with earthquake activity in Japan between the years of 2000 and 2013.

We analyze the contributions from each of each risk model using the methodologies described in the CSEP, and compare their performance with (XXX method and YYY method). Our results indicate that (XXX result, YYY result).
/*terminando com os resultados obtidos e conclusoes alcancadas.*/